\documentclass[final,hyperref={pdfpagelabels=false}]{beamer}

\usepackage[orientation=portrait,size=a1,scale=1.1]{beamerposter}

\mode<presentation>{
    \usetheme{ITR}    
}

%%% fonts & encoding %%%
\usepackage{newcent}
\renewcommand{\familydefault}{\sfdefault}
\usepackage[english]{babel}
%\usepackage[latin1]{inputenc}
%\usepackage[T1]{fontenc}
\usepackage{amsmath,amssymb}

\usepackage{tikz}
\usetikzlibrary{calc}
\usepackage{pgfplots}
\pgfplotsset{compat=1.8}
\usetikzlibrary{pgfplots.groupplots}

%\newcommand {\tbf}[1] {\textbf{#1}}
%\newcommand {\tit}[1] {\textit{#1}}
%\newcommand {\tmd}[1] {\textmd{#1}}
%\newcommand {\trm}[1] {\textrm{#1}}
%\newcommand {\tsc}[1] {\textsc{#1}}
%\newcommand {\tsf}[1] {\textsf{#1}}
%\newcommand {\tsl}[1] {\textsl{#1}}
%\newcommand {\ttt}[1] {\texttt{#1}}
%\newcommand {\tup}[1] {\textup{#1}}

%\newcommand {\mbf}[1] {\mathbf{#1}}
%\newcommand {\mmd}[1] {\mathmd{#1}}
%\newcommand {\mrm}[1] {\mathrm{#1}}
%\newcommand {\msc}[1] {\mathsc{#1}}
%\newcommand {\msf}[1] {\mathsf{#1}}
%\newcommand {\msl}[1] {\mathsl{#1}}
%\newcommand {\mtt}[1] {\mathtt{#1}}
%\newcommand {\mup}[1] {\mathup{#1}}

%some math functions and symbols
\newcommand{\round}[1]{\left\lfloor #1 \right\rceil}
\newcommand{\floor}[1]{\left\lfloor #1 \right\rfloor}
\newcommand{\fracpart}[1]{\left\langle #1 \right\rangle}
\newcommand{\ceil}[1]{\left\lceil #1 \right\rceil}
\newcommand{\reals}{{\mathbb R}}
\newcommand{\ints}{{\mathbb Z}}
\newcommand{\integers}{{\mathbb Z}}
\newcommand{\sign}[1]{\mathtt{sign}(#1)}
\newcommand{\NP}[1]{\operatorname{NearestPt}{#1}}
\newcommand{\NS}[1]{\operatorname{NearestSet}{#1}}
\newcommand{\bres}[1]{\operatorname{Bres}{#1}}
\newcommand{\vol}[1]{\operatorname{vol}{#1}}
\newcommand{\vor}[1]{\operatorname{Vor}{#1}}
\newcommand{\dealias}{\operatorname{dealias}}
\newcommand{\expect}{{\mathbb E}}

\definecolor{LightYellow}{rgb}{1,1,0.75}
\definecolor{LightApricot}{rgb}{1,0.9,0.6}
\definecolor{LightBlue}{rgb}{0.3,0.3,0.9}
\definecolor{LightRed}{rgb}{0.9,0.3,0.3}
\definecolor{LightGreen}{rgb}{0.3,0.9,0.3}
\definecolor{DarkGreen}{rgb}{0,0.5,0}
\definecolor{DarkBlue}{rgb}{0,0,0.6}
\definecolor{BrightRed}{rgb}{0.8,0,0}

%some commonly used underlined and
%hated symbols
\newcommand{\uY}{\ushort{\mbf{Y}}}
\newcommand{\ueY}{\ushort{Y}}
\newcommand{\uy}{\ushort{\mbf{y}}}
\newcommand{\uey}{\ushort{y}}
\newcommand{\ux}{\ushort{\mbf{x}}}
\newcommand{\uex}{\ushort{x}}
\newcommand{\uhx}{\ushort{\mbf{\hat{x}}}}
\newcommand{\uehx}{\ushort{\hat{x}}}

\newcommand {\figwidth} {100mm}
\newcommand {\etal} {\emph{~et~al.}}
\newcommand {\bul} {$\bullet$ }   % bullet
\newcommand {\fig}[1] {Figure~\ref{#1}}   % references Figure x
\newcommand {\imp} {$\Rightarrow$}   % implication symbol (default)
\newcommand {\impt} {$\Rightarrow$}   % implication symbol (text mode)
\newcommand {\impm} {\Rightarrow}   % implication symbol (math mode)
\newcommand {\vect}[1] {\mathbf{#1}} 
\newcommand {\hvect}[1] {\hat{\mathbf{#1}}}
\newcommand {\del} {\partial}
\newcommand {\eqn}[1] {Equation~(\ref{#1})} 
\newcommand {\tab}[1] {Table~\ref{#1}} % references Table x
\newcommand {\half} {\frac{1}{2}} 
\newcommand {\ten}[1] {\times10^{#1}}
\newcommand {\bra}[2] {\mbox{}_{#2}\langle #1 |}
\newcommand {\ket}[2] {| #1 \rangle_{#2}}
\newcommand {\Bra}[2] {\mbox{}_{#2}\left.\left\langle #1 \right.\right|}
\newcommand {\Ket}[2] {\left.\left| #1 \right.\right\rangle_{#2}}
\newcommand {\im} {\mathrm{Im}}
\newcommand {\re} {\mathrm{Re}}
\newcommand {\braket}[4] {\mbox{}_{#3}\langle #1 | #2 \rangle_{#4}} 
\newcommand {\dotprod}[4] {\mbox{}_{#3}\langle #1 | #2 \rangle_{#4}} 
\newcommand {\trace}[1] {\text{tr}\left(#1\right)}

\newcommand{\complex}{{\mathbb C}}
\newcommand{\rationals}{{\mathbb Q}}
\newcommand{\naturals}{{\mathbb N}}
\renewcommand{\leq}{\leqslant}
\renewcommand{\geq}{\geqslant}
%
% The alternatives environment is superceded in the amstex package
%%
% \ncr superceded in amstex
%
%\newcommand{\ncr}[2]{{\setlength{\arraycolsep}{0 mm}
%  \renewcommand{\arraystretch}{0.6} \br{\begin{array}{c} {#1} \\ {#2}
%  \end{array}}}}
%
\newcommand{\refeqn}[1]{\eqref{#1}}
\newcommand{\reffig}[1]{Figure~\ref{#1}}
\newcommand{\reftable}[1]{Table~\ref{#1}}
\newcommand{\refsec}[1]{Section~\ref{#1}}
\newcommand{\refappendix}[1]{Appendix~\ref{#1}}
\newcommand{\refchapter}[1]{Chapter~\ref{#1}}
\newcommand{\sfrac}[2]{{\scriptstyle \frac{#1}{#2}}}
\providecommand{\half}{\tfrac{1}{2}}
\newcommand{\expl}{\text{e}}
\newcommand{\deriv}[2]{\frac{\text{d} #1}{\text{d} #2}}
\newcommand{\pderiv}[2]{\frac{\partial #1}{\partial #2}}
\newcommand{\sgn}{\operatorname{sgn}}

% spell things correctly
\newenvironment{centre}{\begin{center}}{\end{center}}
\newenvironment{itemise}{\begin{itemize}}{\end{itemize}}

%%%%% optional packages
%\usepackage[square,comma,numbers,sort&compress]{natbib}
		% this is the natural sciences bibliography citation
		% style package.  The options here give citations in
		% the text as numbers in square brackets, separated by
		% commas, citations sorted and consecutive citations
		% compressed 
		% output example: [1,4,12-15]
		
\usepackage{cite}
		
%\usepackage[noabbrv]{unitsdef}
\usepackage{units}

\usepackage{booktabs}
		%creates nice looking tables
		
\usepackage{amsmath,amsfonts,amssymb, amsthm} % this is handy for mathematicians and physicists
			      % see http://www.ams.org/tex/amslatex.html		      
		 
\usepackage[vlined, linesnumbered]{algorithm2e}
	%algorithm writing package
	
\usepackage{mathrsfs}
%fancy math script

\usepackage{ushort}
%enable good underlining in math mode

%------------------------------------------------------------
% Theorem like environments
%
%\newtheorem{theorem}{Theorem}
\theoremstyle{plain}
\newtheorem{acknowledgement}{Acknowledgement}
%\newtheorem{algorithm}{Algorithm}
\newtheorem{axiom}{Axiom}
\newtheorem{case}{Case}
\newtheorem{claim}{Claim}
\newtheorem{conclusion}{Conclusion}
\newtheorem{condition}{Condition}
\newtheorem{conjecture}{Conjecture}
%\newtheorem{corollary}{Corollary}
\newtheorem{criterion}{Criterion}
%\newtheorem{definition}{Definition}
%\newtheorem{example}{Example}
\newtheorem{exercise}{Exercise}
%\newtheorem{lemma}{Lemma}
\newtheorem{notation}{Notation}
%\newtheorem{problem}{Problem}
\newtheorem{proposition}{Proposition}
\newtheorem{remark}{Remark}
%\newtheorem{solution}{Solution}
%\newtheorem{summary}{Summary}
%\numberwithin{equation}{section}
%--------------------------------------------------------


\definecolor{darkgreen}{rgb}{0,0.3,0}
\newcommand{\term}[1]{{\color{darkgreen}\textbf{#1}}}

%%% title, author & contact %%%%%%%%%%%%%%%%%%%%%%%%%
\title[]{Fast sparse period estimation}
\author{Robby.~G.~McKilliam, I.~Vaughan.~L.~Clarkson and Barry.~G.~Quinn}
\newcommand{\email}{robby.mckilliam@unisa.edu.au}
\newcommand{\worksupportedby}{Supported under the Australian Government's Australian Space Research Program}
%%%%%%%%%%%%%%%%%%%%%%%%%%%%%%%%%%%%%%%

\begin{document}
%\frame{

%\vspace{-1.4cm}

\begin{columns}[t] % <-- columns with top alignment
\begin{column}{.49\textwidth} % <-- start the first column

\newcommand{\calC}{{\mathcal C}}
\begin{block}{Received signal model}
Receive $L$ noisy $M$-ary phase-shifted-keyed ($M$-PSK) symbols of the form
\[
y_i = a_0 s_i + w_i, \qquad i = 1, \dots, L.
\]
\begin{itemize}
\item $a_0 = \rho_0 e^{j\theta_0}$ is the unknown carrier phase $\theta_0$ and amplitude $\rho_0$,
\item $s_1,\dots,s_L$ are transmitted $M$-PSK symbols,
\item $w_1,\dots,w_L \in \complex$ are i.i.d. circularly symmetric complex random variables representing noise.
\end{itemize}

Interested in estimating $\rho_0$ and $\theta_0$.
\end{block}  

\begin{block}{Least squares estimator}
If all symbols $s_1,\dots,s_L$ are known
\[
\hat{a}_{\text{uc}} = \arg \min_{a \in \complex} \sum_{i = 1}^L \abs{ y_i - a s_i }^2  = \frac{1}{L} \sum_{i = 1}^L y_i s_i^*.
\]
More interested in the practical situation where symbols are not known,
\[
\hat{a} = \arg\min_{a \in \complex} \min_{s_1, \dots, s_L} \sum_{i = 1}^L \abs{ y_i - a s_i }^2.
\]
This estimator can be computed in $O(L\log L)$ operations.
\end{block}  


\begin{block}{Theorem (Almost sure convergence)}

Let $R_i \geq 0$ and $\Phi_i \in [-\pi,\pi)$ be real random variables satisfying
%\begin{equation}\label{eq:RiandPhii}
\vspace{-0.15cm}
\[
R_ie^{j\Phi_i} = 1 + \frac{w_i}{a_0 s_i},
\]
%\end{equation}
and define the continuous function
\[
G(x) = \expect R_1 \cos\sfracpart{ x + \Phi_1}.
\] 
If $G(x)$ is uniquely maximised at $x = 0$ over the interval $[-\tfrac{\pi}{M},\tfrac{\pi}{M})$, then:
\begin{enumerate}
\item \vspace{-0.15cm} $\sfracpart{\hat{\theta} - \theta_0} \rightarrow 0$ almost surely as $L \rightarrow \infty$,
\item \vspace{-0.2cm} $\hat{\rho} \rightarrow \rho_0 G(0)$ almost surely as $L \rightarrow \infty$,
\end{enumerate}
where $\sfracpart{\cdot}$ takes its argument `modulo $\tfrac{2\pi}{M}$' into $[-\tfrac{\pi}{M},\tfrac{\pi}{M})$.
%\[
%\fracpart{x} = x - \tfrac{2\pi}{M}\operatorname{round}\left(\tfrac{M}{2\pi}x\right).
%\]  
\end{block}

\begin{block}{Theorem (Asymptotic normality)}
Let $f(r,\phi)$ be the joint pdf of $R_1$ and $\Phi_1$, and let
\[
g(\phi) = \int_{0}^{\infty} r f(r,\phi) dr.
\]
Put $\hat{\lambda}_L = \sfracpart{\hat{\theta} - \theta_0}$ and $\hat{m}_L = \hat{\rho} - \rho_0 G(0)$. %If $g(\phi)$ is continuous at $\phi = \tfrac{2\pi}{M}k+\tfrac{\pi}{M}$ for each $k = 0, 1, \dots M-1$, then 
The distribution of $(\sqrt{L}\hat{\lambda}_L, \sqrt{L}\hat{m}_L)$ converges to the bivariate normal with zero mean and covariance matrix
\[
\left( \begin{array}{cc} 
H^{-2} A & 0 \\
0 & \rho_0^2 B
\end{array} \right)
\]
as $L \rightarrow \infty$, where
\vspace{-0.15cm}
\[
H = G(0) -  2 \sin\big(\tfrac{\pi}{M}\big) \sum_{k = 0}^{M-1} g\big(\tfrac{2\pi}{M}k + \tfrac{\pi}{M}\big),
\]
\vspace{-0.15cm}
\[
A = \expect R_1^2\sin^2\fracpart{\Phi_1}, \;\;\; B = \expect R_1^2 \cos^2\fracpart{\Phi_1} - G^2(0). 
\]

\end{block}

\end{column}
\begin{column}{.49\textwidth} % <-- start the second column

\begin{block}{Simulations}

\begin{figure}
  \centering 
  \begin{tikzpicture}
    \selectcolormodel{gray} 
    \begin{axis}[font=\footnotesize,xmode=log,ymode=log,height=15cm,width=25cm,xlabel={$N$},ylabel={time (ms)},ylabel style={at={(0.055,0.42)}},xlabel style={at={(0.5,0.1)}}, legend style={draw=none,fill=none,legend pos=north west,cells={anchor=west},font=\footnotesize}]
      \addplot[mark=o,mark options={scale=3}] table {../code/data/PeriodogramBenchmark};
      \addplot[mark=x,mark options={solid,fill=black,scale=3.6}] table {../code/data/LeastSquaresBenchmark};
      \addplot[mark=*,mark options={solid,fill=black,scale=1.5}] table {../code/data/QuantisedPeriodogramq4.0Benchmark};
      \addplot[mark=triangle,mark options={solid,fill=black,scale=2.7}] table {../code/data/QuantisedPeriodogramChirpZq4.0Benchmark};
      %\addplot[mark=square,color=black,mark options={solid,fill=black,scale=0.8}] table {code/data/SLS2novlpBenchmark};
      \addplot[mark=square,color=black,mark options={solid,fill=black,scale=2.4}] table {../code/data/SLS2newBenchmark};
      \legend{Periodogram, Least squares, Fast Fourier transform, Chirp z-transform, SLS-new}
   \end{axis}
  \end{tikzpicture}  
  \caption{Computation time in milliseconds versus $N$ for the periodogram, least squares, and quantized periodogram estimators computed using the chirp z-transform and a single fast Fourier transform.}\label{plot:benchmark}
\end{figure} 



% \begin{figure}
% 	\centering
% 		\includegraphics[width=\linewidth]{../code/data/posterplot-2.mps}
% 		\caption{Phase mean square error for $4$-PSK (QPSK)}
% 		\label{fig:plotphaseM4}
% \end{figure}

% \vspace{0.79cm}
% \begin{figure}
% 	\centering
% 		\includegraphics[width=\linewidth]{../code/data/posterplot-1.mps}
% 		\caption{Amplitude mean square error for $4$-PSK (QPSK)}
% 		\label{fig:plotampM4}
% \end{figure}


\end{block}


\end{column}
\end{columns}

 %}

\end{document}
