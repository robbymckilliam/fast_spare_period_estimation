\documentclass[final,hyperref={pdfpagelabels=false}]{beamer}

\usepackage[orientation=portrait,size=a1,scale=1.1]{beamerposter}

\mode<presentation>{
    \usetheme{ITR}    
}

%%% fonts & encoding %%%
\usepackage{newcent}
\renewcommand{\familydefault}{\sfdefault}
\usepackage[english]{babel}
%\usepackage[latin1]{inputenc}
%\usepackage[T1]{fontenc}
\usepackage{amsmath,amssymb}

\usepackage{tikz}
\usetikzlibrary{calc}
\usepackage{pgfplots}
%\pgfplotsset{compat=1.8}
\usetikzlibrary{pgfplots.groupplots}
\usetikzlibrary{arrows,arrows.new}

%\newcommand {\tbf}[1] {\textbf{#1}}
%\newcommand {\tit}[1] {\textit{#1}}
%\newcommand {\tmd}[1] {\textmd{#1}}
%\newcommand {\trm}[1] {\textrm{#1}}
%\newcommand {\tsc}[1] {\textsc{#1}}
%\newcommand {\tsf}[1] {\textsf{#1}}
%\newcommand {\tsl}[1] {\textsl{#1}}
%\newcommand {\ttt}[1] {\texttt{#1}}
%\newcommand {\tup}[1] {\textup{#1}}

%\newcommand {\mbf}[1] {\mathbf{#1}}
%\newcommand {\mmd}[1] {\mathmd{#1}}
%\newcommand {\mrm}[1] {\mathrm{#1}}
%\newcommand {\msc}[1] {\mathsc{#1}}
%\newcommand {\msf}[1] {\mathsf{#1}}
%\newcommand {\msl}[1] {\mathsl{#1}}
%\newcommand {\mtt}[1] {\mathtt{#1}}
%\newcommand {\mup}[1] {\mathup{#1}}

%some math functions and symbols
\newcommand{\round}[1]{\left\lfloor #1 \right\rceil}
\newcommand{\floor}[1]{\left\lfloor #1 \right\rfloor}
\newcommand{\fracpart}[1]{\left\langle #1 \right\rangle}
\newcommand{\ceil}[1]{\left\lceil #1 \right\rceil}
\newcommand{\reals}{{\mathbb R}}
\newcommand{\ints}{{\mathbb Z}}
\newcommand{\integers}{{\mathbb Z}}
\newcommand{\sign}[1]{\mathtt{sign}(#1)}
\newcommand{\NP}[1]{\operatorname{NearestPt}{#1}}
\newcommand{\NS}[1]{\operatorname{NearestSet}{#1}}
\newcommand{\bres}[1]{\operatorname{Bres}{#1}}
\newcommand{\vol}[1]{\operatorname{vol}{#1}}
\newcommand{\vor}[1]{\operatorname{Vor}{#1}}
\newcommand{\dealias}{\operatorname{dealias}}
\newcommand{\expect}{{\mathbb E}}

\definecolor{LightYellow}{rgb}{1,1,0.75}
\definecolor{LightApricot}{rgb}{1,0.9,0.6}
\definecolor{LightBlue}{rgb}{0.3,0.3,0.9}
\definecolor{LightRed}{rgb}{0.9,0.3,0.3}
\definecolor{LightGreen}{rgb}{0.3,0.9,0.3}
\definecolor{DarkGreen}{rgb}{0,0.5,0}
\definecolor{DarkBlue}{rgb}{0,0,0.6}
\definecolor{BrightRed}{rgb}{0.8,0,0}

%some commonly used underlined and
%hated symbols
\newcommand{\uY}{\ushort{\mbf{Y}}}
\newcommand{\ueY}{\ushort{Y}}
\newcommand{\uy}{\ushort{\mbf{y}}}
\newcommand{\uey}{\ushort{y}}
\newcommand{\ux}{\ushort{\mbf{x}}}
\newcommand{\uex}{\ushort{x}}
\newcommand{\uhx}{\ushort{\mbf{\hat{x}}}}
\newcommand{\uehx}{\ushort{\hat{x}}}

\newcommand {\figwidth} {100mm}
\newcommand {\etal} {\emph{~et~al.}}
\newcommand {\bul} {$\bullet$ }   % bullet
\newcommand {\fig}[1] {Figure~\ref{#1}}   % references Figure x
\newcommand {\imp} {$\Rightarrow$}   % implication symbol (default)
\newcommand {\impt} {$\Rightarrow$}   % implication symbol (text mode)
\newcommand {\impm} {\Rightarrow}   % implication symbol (math mode)
\newcommand {\vect}[1] {\mathbf{#1}} 
\newcommand {\hvect}[1] {\hat{\mathbf{#1}}}
\newcommand {\del} {\partial}
\newcommand {\eqn}[1] {Equation~(\ref{#1})} 
\newcommand {\tab}[1] {Table~\ref{#1}} % references Table x
\newcommand {\half} {\frac{1}{2}} 
\newcommand {\ten}[1] {\times10^{#1}}
\newcommand {\bra}[2] {\mbox{}_{#2}\langle #1 |}
\newcommand {\ket}[2] {| #1 \rangle_{#2}}
\newcommand {\Bra}[2] {\mbox{}_{#2}\left.\left\langle #1 \right.\right|}
\newcommand {\Ket}[2] {\left.\left| #1 \right.\right\rangle_{#2}}
\newcommand {\im} {\mathrm{Im}}
\newcommand {\re} {\mathrm{Re}}
\newcommand {\braket}[4] {\mbox{}_{#3}\langle #1 | #2 \rangle_{#4}} 
\newcommand {\dotprod}[4] {\mbox{}_{#3}\langle #1 | #2 \rangle_{#4}} 
\newcommand {\trace}[1] {\text{tr}\left(#1\right)}

\newcommand{\complex}{{\mathbb C}}
\newcommand{\rationals}{{\mathbb Q}}
\newcommand{\naturals}{{\mathbb N}}
\renewcommand{\leq}{\leqslant}
\renewcommand{\geq}{\geqslant}
%
% The alternatives environment is superceded in the amstex package
%%
% \ncr superceded in amstex
%
%\newcommand{\ncr}[2]{{\setlength{\arraycolsep}{0 mm}
%  \renewcommand{\arraystretch}{0.6} \br{\begin{array}{c} {#1} \\ {#2}
%  \end{array}}}}
%
\newcommand{\refeqn}[1]{\eqref{#1}}
\newcommand{\reffig}[1]{Figure~\ref{#1}}
\newcommand{\reftable}[1]{Table~\ref{#1}}
\newcommand{\refsec}[1]{Section~\ref{#1}}
\newcommand{\refappendix}[1]{Appendix~\ref{#1}}
\newcommand{\refchapter}[1]{Chapter~\ref{#1}}
\newcommand{\sfrac}[2]{{\scriptstyle \frac{#1}{#2}}}
\providecommand{\half}{\tfrac{1}{2}}
\newcommand{\expl}{\text{e}}
\newcommand{\deriv}[2]{\frac{\text{d} #1}{\text{d} #2}}
\newcommand{\pderiv}[2]{\frac{\partial #1}{\partial #2}}
\newcommand{\sgn}{\operatorname{sgn}}

% spell things correctly
\newenvironment{centre}{\begin{center}}{\end{center}}
\newenvironment{itemise}{\begin{itemize}}{\end{itemize}}

%%%%% optional packages
%\usepackage[square,comma,numbers,sort&compress]{natbib}
		% this is the natural sciences bibliography citation
		% style package.  The options here give citations in
		% the text as numbers in square brackets, separated by
		% commas, citations sorted and consecutive citations
		% compressed 
		% output example: [1,4,12-15]
		
\usepackage{cite}
		
%\usepackage[noabbrv]{unitsdef}
\usepackage{units}

\usepackage{booktabs}
		%creates nice looking tables
		
\usepackage{amsmath,amsfonts,amssymb, amsthm} % this is handy for mathematicians and physicists
			      % see http://www.ams.org/tex/amslatex.html		      
		 
\usepackage[vlined, linesnumbered]{algorithm2e}
	%algorithm writing package
	
\usepackage{mathrsfs}
%fancy math script

\usepackage{ushort}
%enable good underlining in math mode

%------------------------------------------------------------
% Theorem like environments
%
%\newtheorem{theorem}{Theorem}
\theoremstyle{plain}
\newtheorem{acknowledgement}{Acknowledgement}
%\newtheorem{algorithm}{Algorithm}
\newtheorem{axiom}{Axiom}
\newtheorem{case}{Case}
\newtheorem{claim}{Claim}
\newtheorem{conclusion}{Conclusion}
\newtheorem{condition}{Condition}
\newtheorem{conjecture}{Conjecture}
%\newtheorem{corollary}{Corollary}
\newtheorem{criterion}{Criterion}
%\newtheorem{definition}{Definition}
%\newtheorem{example}{Example}
\newtheorem{exercise}{Exercise}
%\newtheorem{lemma}{Lemma}
\newtheorem{notation}{Notation}
%\newtheorem{problem}{Problem}
\newtheorem{proposition}{Proposition}
\newtheorem{remark}{Remark}
%\newtheorem{solution}{Solution}
%\newtheorem{summary}{Summary}
%\numberwithin{equation}{section}
%--------------------------------------------------------


\definecolor{darkgreen}{rgb}{0,0.3,0}
\newcommand{\term}[1]{{\color{darkgreen}\textbf{#1}}}

%%% title, author & contact %%%%%%%%%%%%%%%%%%%%%%%%%
\title[]{Fast sparse period estimation}
\author{Robby~G.~McKilliam, I.~Vaughan~L.~Clarkson \\ and Barry~G.~Quinn}
\newcommand{\email}{robby.mckilliam@unisa.edu.au}
\newcommand{\worksupportedby}{Supported by Australian Research Council Discovery Project DP130102228}
%%%%%%%%%%%%%%%%%%%%%%%%%%%%%%%%%%%%%%%

\begin{document}
%\frame{

%\vspace{-1.4cm}

\begin{columns}[t] % <-- columns with top alignment
\begin{column}{.49\textwidth} % <-- start the first column

\newcommand{\Tmin}{T_{\text{min}}}
\newcommand{\Tmax}{2T_{\text{min}}}

\begin{block}{Sparse noisy period estimation}

Observations of the form
\[
y_n = T_0 s_n + \theta_0 + w_n, \qquad n = 1,\dots,N
\]
where
\begin{itemize}
\item $T_0 \in (\Tmin, \Tmax) \subset \reals$ is the unknown period,
\item $\theta_0 \in \reals$ is an unknown phase,
\item $w_1,\dots,w_N \in \reals$ represent noise,
\item $s_1,\dots,s_N \in \ints$ represent the subset of periods observed.
\end{itemize}
Aim to estimate $T_0$.

{
\def\vertgap{2}
\def\ph{0.4}
\def\T{1.1}

\newcommand{\raxis}{\draw[-> new,arrow head=2mm] (-0.25,0) -- (8,0) node[above] {$\reals$}; \draw (0,-0.06)-- node[below] {$0$} (0,0.06) }
\newcommand{\pulse}[1]{ \draw[-latex new,arrow head=4mm] (#1,0) -- (#1,1) }
\newcommand{\pulsewithnode}[2]{ \draw[-latex new,arrow head=4mm] (#1,0) -- node[right] {#2} (#1,1) }
\begin{figure}[t]
	\centering
\begin{tikzpicture}[font=\small,scale=3]
    %\draw[very thin,color=gray] (-0.1,-1.1) grid (3.9,3.9);
\begin{scope}
    \raxis;
    \foreach \k in {0,1,...,6} { \k, \pulse{\T*\k+\ph}; }
    \draw[< new-> new, arrow head=2mm] (\T*2+\ph,1.2) -- node[above] {$T_0$} (\T*3+\ph,1.2);
    \draw[< new-> new, arrow head=2mm] (0.0,1.2) -- node[above] {$\theta_0$} (\ph,1.2); 
%    \draw[->] (-0.25,-2*\vertgap) -- (8,-2*\vertgap) node[above] {$\reals$};
\end{scope}
\begin{scope}[yshift=-\vertgap cm]
  \raxis;
  \node[below] at (\T*1+\ph-0.3,0) {$y_1$}; \pulsewithnode{\T*1+\ph-0.3}{$s_1=1$}; \draw[-> new arrow head=2mm] (\T*1+\ph,1.2) -- node[above] {$w_1$} (\T*1+\ph-0.3,1.2); 
  \node[below] at (\T*3+\ph+0.5,0) {$y_2$}; \pulsewithnode{\T*3+\ph+0.5}{$s_2=3$}; \draw[-> new,arrow head=2mm] (\T*3+\ph,1.2) -- node[above] {$w_2$} (\T*3+\ph+0.5,1.2);
  \node[below] at (\T*6+\ph-0.4,0) {$y_3$}; \pulsewithnode{\T*6+\ph-0.4}{$s_3=6$};\draw[-> new,arrow head=2mm] (\T*6+\ph,1.2) -- node[above] {$w_3$} (\T*6+\ph-0.4,1.2);
\end{scope}
\end{tikzpicture} 
%		\includegraphics{figs/figures-1.mps}
	%	\caption{Sparse noisy observations of a periodic sequence.}
		\label{fig_stat_model}
\end{figure}
}


\end{block}  

\begin{block}{Applications}
\begin{itemize}
\item Pulse interval estimation in electronic support.%~\cite{Clarkson_thesis,clarkson_estimate_period_pulse_train_1996,Hauochan_pri_2012}.
\item Synchronisation for telecommunications.%~\cite{Fogel1989_bit_synch_zero_crossings,Sidiropoulos2005,5621928}.
\item Spike interval estimation in neurology.%~\cite{Brillinger_spike_trains_1988}.
\item Detection of anomalies in internet traffic.%~\cite{He_detecting_periodic_patterns_in_internet_2006,5585849,5947313}.
\end{itemize}
Connected with Diophantine approximation, lattice reduction, and the study of prime numbers.%~\cite{Clarkson_thesis,CaseySadler_primes_2013}.
\end{block}

\newcommand{\abs}[1]{\left\vert #1 \right\vert}
\newcommand{\sabs}[1]{\vert #1 \vert}
\newcommand{\babs}[1]{\big\vert #1 \big\vert}


\begin{block}{The periodogram estimator}
An accurate estimator is $\widehat{T} = 1/\widehat{f}$ where $\widehat{f}$ maximises the \emph{periodogram}
\[
I_y(f) = \babs{ \sum_{n=1}^N e^{ 2\pi j f y_n} }^2.
\]
over the interval $(1/\Tmax, 1/\Tmin)$.  
\begin{itemize}
\item Requires $O(N^2)$ operations to compute. 
\item Observations $y_1,\dots,y_N$ occur in the exponent preventing use of fast transforms.
\end{itemize}
\end{block}

\newcommand{\ellmin}{\ell_{\text{min}}}
\newcommand{\ellmax}{\ell_{\text{max}}}

\begin{block}{A fast approximation}

Compute quantised observations
\[
z_n =  \ell_n/q, \qquad n = 1,\dots,N
\] 
where $\ell_n = \round{qy_n} \in \ints$ and maximise
\[
I_z(f) = \babs{ \sum_{n=1}^N e^{ 2\pi j f z_n} }^2 = \babs{ \sum_{i = \ellmin}^{\ellmax} e^{2\pi j f i/q} b_i }^2
\]
where $\ellmin$ and $\ellmax$ are the minimum and maximum of $\ell_1,\dots,\ell_n$ and
\[
b_i= \#\{ n \mid \ell_n = i, n = 1,\dots, N\}
\]
where $\#$ indicates the number of elements in a set.

\begin{itemize}
\item Can be computed using $O(N\log N)$ operations by fast Fourier transform or chirp-z transform.
\end{itemize}

\end{block}

\end{column}
\begin{column}{.49\textwidth} % <-- start the second column

\begin{block}{Simulations}

\begin{figure}[p] 
  \centering 
  \begin{tikzpicture} 
\begin{groupplot}[
    group style={ 
        group name=my plots,
        group size=1 by 2,
        %ylabels at=edge left,
        xlabels at=edge bottom,
        xticklabels at=edge bottom,
        vertical sep=0pt
    },
    ylabel style={at={(0.085,0.4)}},
    footnotesize,
    width=25cm,
    height=16cm,
    tickpos=left,
    %ytick align=outside,
    %xtick align=outside,
    xmode=log,
    ymode=log,
    xmin=11e-5,
    xmax=0.4
%    enlarge x limits=false 
]

\nextgroupplot[
ylabel={MSE},
ymin=2e-10
]

      \addplot[mark=none,color=black,dashed,thick] table {../code/data/NormalisedLLSCLTN30geom10}; 

            \addplot[mark=*,mark options={scale=1.5},color=cyan] table {../code/data/QuantisedPeriodogramFFTN30q1.5geom10};
      \addplot[mark=square*,mark options={scale=2.5},color=green] table {../code/data/QuantisedPeriodogramFFTN30q5.0geom10};
\addplot[mark=triangle*,mark options={scale=3},color=blue] table {../code/data/SLS2newN30geom10};
\addplot[mark=none,color=black,thick] table {../code/data/PeriodogramCLTN30geom10}; 
 \addplot[mark=o,mark options={scale=2.5},color=black, thick] table {../code/data/NormalisedSamplingLLSN30geom10}; 
      \addplot[mark=*,mark options={scale=1.75},color=red] table {../code/data/PeriodogramN30geom10};
%      
  %     \addplot[mark=square,only marks,color=black,mark options={solid,fill=black,scale=0.8}] table {code/data/SLS2novlpN30geom10};
       
%      \addplot[mark=none,dashed,mark options={solid,fill=black,scale=1}] table {code/data/QuantisedPeriodogramFFTN30q2};
 %     \addplot[mark=none,color=black] table {code/data/PeriodogramCLTN200};
 %     \addplot[mark=none,color=black,dashed] table {code/data/NormalisedLLSCLTN200}; 
 %     \addplot[mark=*,only marks,color=black,mark options={solid,fill=black,scale=0.5}] table {code/data/PeriodogramN200};
 %     \addplot[mark=x,only marks,color=black,mark options={solid,fill=black,scale=1}] table {code/data/QuantisedPeriodogramFFTN200q3};
 %     \addplot[mark=o,only marks,color=black,mark options={solid,fill=black,scale=1}] table {code/data/QuantisedPeriodogramFFTN200q10};
%      \addplot[mark=x,only marks,color=black,mark options={solid,fill=black,scale=0.8}] table {code/data/NormalisedSamplingLLSN200}; 
       \node[font=\footnotesize] at (axis cs:0.2,4e-10) {$N=30$};

\nextgroupplot[
xlabel={$\sigma^2$},
xlabel style={at={(0.5,0.13)}},
ylabel={MSE},
ymin=2e-15,
ymax=8e-1,
legend style={
  legend columns=1,
  draw=none,
  %fill=none,
  legend pos=north west,
  %at={(0.75,1)},
  cells={anchor=west},
  font=\footnotesize
}
%ytickten={-1,-3,-5,-7,-9,-11,-13}
]

\addplot[mark=*,mark options={scale=1.5},color=cyan] table {../code/data/QuantisedPeriodogramFFTN1200q1.5geom10};
 \addplot[mark=square*,mark options={scale=2.5},color=green] table {../code/data/QuantisedPeriodogramFFTN1200q5.0geom10};
       \addplot[mark=triangle*,mark options={scale=3},color=blue] table {../code/data/SLS2newN1200geom10};
\addplot[mark=none,color=black,dashed] table {../code/data/NormalisedLLSCLTN1200geom10};
\addplot[mark=none,color=black] table {../code/data/PeriodogramCLTN1200geom10};
       \addplot[mark=o,mark options={scale=2.5},color=black, thick] table {../code/data/NormalisedSamplingLLSN1200geom10};
       \addplot[mark=*,mark options={scale=1.75},color=red] table {../code/data/PeriodogramN1200geom10};
 %      \addplot[mark=square,only marks,color=black,mark options={solid,fill=black,scale=0.8}] table {code/data/SLS2novlpN1100geom10};
      %\addplot[mark=x,only marks,color=black,mark options={solid,fill=black,scale=1}] table {code/data/QuantisedPeriodogramFFTN1100q3};
 %     \legend{Periodogram Theory, Least squares theory, Quantised periodogram, Periodogram, Least squares}
     %\legend{Periodogram theory, $q=\tfrac{3}{2}T_{\text{max}}$, $q=4T_{\text{max}}$, Periodogram, Least squares, SLS2-adj}
     \node[font=\footnotesize] at (axis cs:0.32,2e-15) [anchor=south east] {$N=1200$};
   %  \node[rotate=16,font=\footnotesize] at (axis cs:7e-4,1.5e-14) {$N=1200$}; 

\legend{Quantised periodogram $q=1.5$, Quantised periodogram $q=5$,Separable line search,Least squares Theory,Periodogram Theory, Least squares,Periodogram}

\end{groupplot}

  \end{tikzpicture}  
  \caption{Mean square error versus noise variance $\sigma^2$.  Trials are generated so that only $10\%$ of periods are observed on average. }\label{plot:multipleN}
\end{figure} 


\begin{figure}
  \centering 
  \begin{tikzpicture}
%    \selectcolormodel{gray} 
    \begin{axis}[font=\footnotesize,xmode=log,ymode=log,height=15.25cm,width=25cm,xlabel={$N$},ylabel={time (ms)},ylabel style={at={(0.085,0.42)}},xlabel style={at={(0.5,0.13)}}, legend style={draw=none,fill=none,legend pos=north west,cells={anchor=west},font=\footnotesize},xmin=7,xmax=36000]
      \addplot[mark=*,mark options={scale=1.75},color=red] table {../code/data/PeriodogramBenchmark};
      \addplot[mark=o,mark options={scale=2.5},color=black, thick] table {../code/data/LeastSquaresBenchmark};
      %\addplot[mark=triangle*,mark options={scale=3},color=cyan] table {../code/data/QuantisedPeriodogramChirpZq4.0Benchmark};
      %\addplot[mark=square,color=black,mark options={solid,fill=black,scale=0.8}] table {code/data/SLS2novlpBenchmark};
      \addplot[mark=triangle*,mark options={scale=3},color=blue] table {../code/data/SLS2newBenchmark};
      \addplot[mark=square*,mark options={scale=2.5},color=green] table {../code/data/QuantisedPeriodogramq4.0Benchmark};
      %\legend{Periodogram, Least squares, Fast Fourier transform, Chirp z-transform, SLS-new}
      \legend{Periodogram, Least squares, Separable line search, Quantised periodogram $q=5$}
   \end{axis}
  \end{tikzpicture}  
  \caption{Computation time in milliseconds versus data length $N$.}\label{plot:benchmark}
\end{figure} 



% \begin{figure}
% 	\centering
% 		\includegraphics[width=\linewidth]{../code/data/posterplot-2.mps}
% 		\caption{Phase mean square error for $4$-PSK (QPSK)}
% 		\label{fig:plotphaseM4}
% \end{figure}

% \vspace{0.79cm}
% \begin{figure}
% 	\centering
% 		\includegraphics[width=\linewidth]{../code/data/posterplot-1.mps}
% 		\caption{Amplitude mean square error for $4$-PSK (QPSK)}
% 		\label{fig:plotampM4}
% \end{figure}


\end{block}

\begin{block}{References}

\nocite{Fogel1989_bit_synch_zero_crossings,Sidiropoulos2005,Clarkson2007,Quinn_sparse_noisy_SSP_2012,Quinn20013asilomar_period_est}

\bibliographystyle{../IEEEbib}
\footnotesize
\bibliography{../bib}

\end{block}


\end{column}
\end{columns}

 %}

\end{document}
