\documentclass[a4paper,10pt]{article}
%\documentclass[draftcls, onecolumn, 11pt]{IEEEtran}
%\documentclass[journal]{IEEEtran}
 
\usepackage{../mathbf-abbrevs}
%\newcommand {\tbf}[1] {\textbf{#1}}
%\newcommand {\tit}[1] {\textit{#1}}
%\newcommand {\tmd}[1] {\textmd{#1}}
%\newcommand {\trm}[1] {\textrm{#1}}
%\newcommand {\tsc}[1] {\textsc{#1}}
%\newcommand {\tsf}[1] {\textsf{#1}}
%\newcommand {\tsl}[1] {\textsl{#1}}
%\newcommand {\ttt}[1] {\texttt{#1}}
%\newcommand {\tup}[1] {\textup{#1}}
%
%\newcommand {\mbf}[1] {\mathbf{#1}}
%\newcommand {\mmd}[1] {\mathmd{#1}}
%\newcommand {\mrm}[1] {\mathrm{#1}}
%\newcommand {\msc}[1] {\mathsc{#1}}
%\newcommand {\msf}[1] {\mathsf{#1}}
%\newcommand {\msl}[1] {\mathsl{#1}}
%\newcommand {\mtt}[1] {\mathtt{#1}}
%\newcommand {\mup}[1] {\mathup{#1}}

%some math functions and symbols
\newcommand{\reals}{{\mathbb R}}
\newcommand{\expect}{{\mathbb E}}
\newcommand{\indicator}{{\mathbf 1}}
\newcommand{\ints}{{\mathbb Z}}
\newcommand{\complex}{{\mathbb C}}
\newcommand{\integers}{{\mathbb Z}}
\newcommand{\sign}[1]{\mathtt{sign}\left( #1 \right)}
\newcommand{\NP}{\operatorname{NPt}}
\newcommand{\erf}{\operatorname{erf}}
\newcommand{\NS}{\operatorname{NearestSet}}
\newcommand{\bres}{\operatorname{Bres}}
\newcommand{\vol}{\operatorname{vol}}
\newcommand{\vor}{\operatorname{Vor}}
%\newcommand{\re}{\operatorname{Re}}
%\newcommand{\im}{\operatorname{Im}}




\newcommand{\term}{\emph}
\newcommand{\var}{\operatorname{var}}
\newcommand{\covar}{\operatorname{cov}}
%\newcommand{\prob}{{\mathbb P}}
\newcommand{\prob}{{\operatorname{Pr}}}

%distribution fucntions
\newcommand{\projnorm}{\operatorname{ProjectedNormal}}
\newcommand{\vonmises}{\operatorname{VonMises}}
\newcommand{\wrapnorm}{\operatorname{WrappedNormal}}
\newcommand{\wrapunif}{\operatorname{WrappedUniform}}

\newcommand{\selectindicies}{\operatorname{selectindices}}
\newcommand{\sortindicies}{\operatorname{sortindices}}
\newcommand{\largest}{\operatorname{largest}}
\newcommand{\quickpartition}{\operatorname{quickpartition}}
\newcommand{\quickpartitiontwo}{\operatorname{quickpartition2}}

%some commonly used underlined and
%hated symbols
\newcommand{\uY}{\ushort{\mbf{Y}}}
\newcommand{\ueY}{\ushort{Y}}
\newcommand{\uy}{\ushort{\mbf{y}}}
\newcommand{\uey}{\ushort{y}}
\newcommand{\ux}{\ushort{\mbf{x}}}
\newcommand{\uex}{\ushort{x}}
\newcommand{\uhx}{\ushort{\mbf{\hat{x}}}}
\newcommand{\uehx}{\ushort{\hat{x}}}

% Brackets
\newcommand{\br}[1]{{\left( #1 \right)}}
\newcommand{\sqbr}[1]{{\left[ #1 \right]}}
\newcommand{\cubr}[1]{{\left\{ #1 \right\}}}
\newcommand{\abr}[1]{\left< #1 \right>}
\newcommand{\abs}[1]{\left\vert #1 \right\vert}
\newcommand{\sabs}[1]{\vert #1 \vert}
\newcommand{\babs}[1]{\big\vert #1 \big\vert}
\newcommand{\sfloor}[1]{{\lfloor #1 \rfloor}}
\newcommand{\floor}[1]{{\left\lfloor #1 \right\rfloor}}
\newcommand{\ceiling}[1]{{\left\lceil #1 \right\rceil}}
\newcommand{\ceil}[1]{\left\lceil #1 \right\rceil}
\newcommand{\round}[1]{{\left\lfloor #1 \right\rceil}}
\newcommand{\magn}[1]{\left\| #1 \right\|}
\newcommand{\fracpart}[1]{\left\langle #1 \right\rangle}
\newcommand{\sfracpart}[1]{\langle #1 \rangle}


% Referencing
\newcommand{\refeqn}[1]{\eqref{#1}}
\newcommand{\reffig}[1]{Figure~\ref{#1}}
\newcommand{\reftable}[1]{Table~\ref{#1}}
\newcommand{\refsec}[1]{Section~\ref{#1}}
\newcommand{\refappendix}[1]{Appendix~\ref{#1}}
\newcommand{\refchapter}[1]{Chapter~\ref{#1}}

\newcommand {\figwidth} {100mm}
\newcommand {\Ref}[1] {Reference~\cite{#1}}
\newcommand {\Sec}[1] {Section~\ref{#1}}
\newcommand {\App}[1] {Appendix~\ref{#1}}
\newcommand {\Chap}[1] {Chapter~\ref{#1}}
\newcommand {\Lem}[1] {Lemma~\ref{#1}}
\newcommand {\Thm}[1] {Theorem~\ref{#1}}
\newcommand {\Cor}[1] {Corollary~\ref{#1}}
\newcommand {\Alg}[1] {Algorithm~\ref{#1}}
\newcommand {\etal} {\emph{~et~al.}}
\newcommand {\bul} {$\bullet$ }   % bullet
\newcommand {\fig}[1] {Figure~\ref{#1}}   % references Figure x
\newcommand {\imp} {$\Rightarrow$}   % implication symbol (default)
\newcommand {\impt} {$\Rightarrow$}   % implication symbol (text mode)
\newcommand {\impm} {\Rightarrow}   % implication symbol (math mode)
\newcommand {\vect}[1] {\mathbf{#1}} 
\newcommand {\hvect}[1] {\hat{\mathbf{#1}}}
\newcommand {\del} {\partial}
\newcommand {\eqn}[1] {Equation~(\ref{#1})} 
\newcommand {\tab}[1] {Table~\ref{#1}} % references Table x
\newcommand {\half} {\frac{1}{2}} 
\newcommand {\ten}[1] {\times10^{#1}}
\newcommand {\bra}[2] {\mbox{}_{#2}\langle #1 |}
\newcommand {\ket}[2] {| #1 \rangle_{#2}}
\newcommand {\Bra}[2] {\mbox{}_{#2}\left.\left\langle #1 \right.\right|}
\newcommand {\Ket}[2] {\left.\left| #1 \right.\right\rangle_{#2}}
\newcommand {\im} {\mathrm{Im}}
\newcommand {\re} {\mathrm{Re}}
\newcommand {\braket}[4] {\mbox{}_{#3}\langle #1 | #2 \rangle_{#4}} 
\newcommand{\dotprod}[2]{ \left\langle #1 , #2 \right\rangle}
\newcommand {\trace}[1] {\text{tr}\left(#1\right)}

% spell things correctly
\newenvironment{centre}{\begin{center}}{\end{center}}
\newenvironment{itemise}{\begin{itemize}}{\end{itemize}}

%%%%% set up the bibliography style
\bibliographystyle{IEEEbib}
%\bibliographystyle{uqthesis}  % uqthesis bibliography style file, made
			      % with makebst

%%%%% optional packages
\usepackage[square,comma,numbers,sort&compress]{natbib}
		% this is the natural sciences bibliography citation
		% style package.  The options here give citations in
		% the text as numbers in square brackets, separated by
		% commas, citations sorted and consecutive citations
		% compressed 
		% output example: [1,4,12-15]

%\usepackage{cite}		
			
\usepackage{units}
	%nice looking units
		
\usepackage{booktabs}
		%creates nice looking tables
		
\usepackage{ifpdf}
\ifpdf
  \usepackage[pdftex]{graphicx}
  %\usepackage{thumbpdf}
  %\usepackage[naturalnames]{hyperref}
\else
	\usepackage{graphicx}% standard graphics package for inclusion of
		      % images and eps files into LaTeX document
\fi

\usepackage{amsmath,amsfonts,amssymb, amsthm, bm} % this is handy for mathematicians and physicists
			      % see http://www.ams.org/tex/amslatex.html

		 
\usepackage[vlined, linesnumbered]{algorithm2e}
	%algorithm writing package
	
\usepackage{mathrsfs}
%fancy math script

%\usepackage{ushort}
%enable good underlining in math mode

%------------------------------------------------------------
% Theorem like environments
%
\newtheorem{theorem}{Theorem}
%\theoremstyle{plain}
\newtheorem{acknowledgement}{Acknowledgement}
%\newtheorem{algorithm}{Algorithm}
\newtheorem{axiom}{Axiom}
\newtheorem{case}{Case}
\newtheorem{claim}{Claim}
\newtheorem{conclusion}{Conclusion}
\newtheorem{condition}{Condition}
\newtheorem{conjecture}{Conjecture}
\newtheorem{corollary}{Corollary}
\newtheorem{criterion}{Criterion}
\newtheorem{definition}{Definition}
\newtheorem{example}{Example}
\newtheorem{exercise}{Exercise}
\newtheorem{lemma}{Lemma}
\newtheorem{notation}{Notation}
\newtheorem{problem}{Problem}
\newtheorem{proposition}{Proposition}
\newtheorem{property}{Property}
\newtheorem{remark}{Remark}
\newtheorem{solution}{Solution}
\newtheorem{summary}{Summary}
%\numberwithin{equation}{section}
%--------------------------------------------------------


\usepackage{zref-xr,zref-user}
\zexternaldocument*{paper}

\usepackage[left=2cm,right=2cm,top=2cm,bottom=2cm]{geometry}

\usepackage{amsmath,amsfonts,amssymb, amsthm, bm}

%\usepackage[square,comma,numbers,sort&compress]{natbib}

\usepackage{color}
\newcommand{\comment}[1]{\textcolor{red}{#1}}

\newcommand{\sgn}{\operatorname{sgn}}
\newcommand{\sinc}{\operatorname{sinc}}
\newcommand{\rect}[1]{\operatorname{rect}\left(#1\right)}

%opening
\title{Fast sparse period estimation: Reviewer responses}
\author{R.~G.~McKilliam, I.~V.~L.~Clarkson and B.~G.~Quinn 
%\thanks{}
}

\renewcommand{\theenumii}{(\alph{enumii})}
\renewcommand{\labelenumii}{\theenumii}

\begin{document}

\maketitle


\section*{Reviewer 1}\label{sec:reviewer-1}

\begin{enumerate}

\item\textbf{Comment:}
This paper proposes a new fast method for estimating the period of sparse and noisy
observations of a point process. The new idea is to used the periodogram of quantized
observations that can be computed faster than the standard periodogram by using
a chirp z-transform or a fast Fourier transform. The reviewer is not convinced that
there is enough material deserving publication in the IEEE signal processing letters.
Moreover, there are several important comments that should be addressed before a
possible publication of this letter.
\\
\textbf{Response:}
BLERG: Do I want to say something it being simple, but novel, and potentially really useful!

\item\textbf{Comment:}
The main reviewer concern is about the choice of the quantization parameter
q that should be explained more carefully. It is not sufficient to display two sets
of curves obtained for two different values of q and to choose the best one without
explanations. For instance, it is not clear whether the value of q chosen in this paper
will be appropriate for other noise distributions or for other ways of generating the
integers sn. This point should be addressed carefully before a possible resubmission
of the paper. The other reviewer comments are summarized below.
\\
\textbf{Response:}
BLERG: Run some extra simulations for them

\item\textbf{Comment:}
p. 3, the authors mention “Sadler and Casey [24]... However, their estimator
requires far larger transforms and produces less accurate results than the estimator
described here (see Section VI)". Unfortunately, there is nothing in
Section VI (which is the conclusion) supporting this claim. The authors should
compare their results in term of accuracy and of computational complexity to
the results of [24]. Including the Cramér-Rao bounds for the estimation of T0
(as in [24]) would also be appreciated.
\\
\textbf{Response:}
BLERG: 


\item\textbf{Comment:}
p. 7, the authors indicate that they have used the Newton-Raphson (NR)
method. More details should be provided here. For instance, what is the exact
NR scheme used in this paper (there are several options)? How have the
authors selected the total number of iterations? Is the quantized periodogram
just used to initialize a Newton-Raphson procedure? Do the MSEs displayed
in Fig. 3 correspond to the estimator obtained after applying the NR algorithm?
Also, do the results displayed in Fig. 2 correspond to the total time
required to compute the estimator (including the NR procedure) or just to the
initialization of the NR procedure?
\\
\textbf{Response:}
BLERG: 


\item\textbf{Comment:}
p. 5, in the definition of $v_m$, $b_{m+\ell_1}$ should be replaced by $b_{m+\ell_{\text{min}}}$.
\\
\textbf{Response:}
Thankyou for spotting this error.  A similar error was fixed in equation for $I_z$ directly above this. 


\item\textbf{Comment:}
p. 6, ``each of length $L+K-1$'' might be replaced by ``each of length $M = L + K - 1$"
\\
\textbf{Response:}
We have not done this.  The parameter $M$ is reserved for the length of the FFT used for the single Fourier transform method.  We feel it will lead to confusion if $M$ is also used for the length of the transforms required for the chirp-z transform.


\item\textbf{Comment:}
p. 8, $n = 2,\dots,N$ should be $n = 1, \dots, N - 1$
\\
\textbf{Response:}



\item\textbf{Comment:}
p. 8, it is not completely clear to identify the curves associated with $N = 30$ and $N = 1200$. Can the authors display 4 curves associated with $\mu= 1$, $\mu = 10$, $N = 30$ and $N = 1200$?
\\
\textbf{Response:}
BLERG: 


\item\textbf{Comment:}
Typos: p. 2, “anomolies" should be “anomalies", p. 7, “quanitised" should be
“quantized".
\\
\textbf{Response:}
BLERG: 


\item\textbf{Comment:}
Is it necessary to include 11 self citations in this letter? Removing some of
these references would allow to save some space, which could be used to better
explain some points mentioned above.
\\
\textbf{Response:}
IEEE Signal Processing Letters allows a 5th page that can contain only references.  Therefore, we cannot save space by removing references.   We have removed reference [29] from the original manuscript because the material in that paper is also covered in [27].  We feel that all 36 other references are strongly relevant and that the paper would be weakened without them.

\end{enumerate}



\section*{Reviewer 2}


\begin{enumerate}

\item\textbf{Comment:}
 The identifiability claim at the beginning of sec. II is not obvious and a reference is mandatory. Indeed, referring to eq. (2), let us assume that $s_n = 2 i_n$ where $i_n$ are all non-negative integer. In this case the point process is periodic with period $2 T_0$. This referee does not believe that maximizing the periodogram over $(f_{\text{min}},f_{\text{max}})$ will identify the true frequency (or period). LS methods might have the capabilities to identify the period, but, in any case the claim of idenitifiability needs to be further supported.
\\
\textbf{Response:}


\item\textbf{Comment:}
 Is the equality in the second (un-numbered) equation of page 2 correct?
\\
\textbf{Response:}

\item\textbf{Comment:}
Newton-Rapson iteration to further improve the frequency estimation is advocated in Sec. II. Can a reference supporting this statement be added?
\\
\textbf{Response:}

\item\textbf{Comment:}
Referring to Fig. 3, in case $N=30$, the novel method incurs in a significant loss of performance with respect to classical LS and periodogram for low-signal-to-noise ratios. In this scenario (low $N$ and low signal-to-noise-ration) relevant for real applications?
\\
\textbf{Response:}

\end{enumerate}



\bibstyle{IEEEtran}
\bibliography{bib}

\end{document}





















