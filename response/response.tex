\documentclass[a4paper,10pt]{article}
%\documentclass[draftcls, onecolumn, 11pt]{IEEEtran}
%\documentclass[journal]{IEEEtran}
 
\usepackage{../mathbf-abbrevs}
%\newcommand {\tbf}[1] {\textbf{#1}}
%\newcommand {\tit}[1] {\textit{#1}}
%\newcommand {\tmd}[1] {\textmd{#1}}
%\newcommand {\trm}[1] {\textrm{#1}}
%\newcommand {\tsc}[1] {\textsc{#1}}
%\newcommand {\tsf}[1] {\textsf{#1}}
%\newcommand {\tsl}[1] {\textsl{#1}}
%\newcommand {\ttt}[1] {\texttt{#1}}
%\newcommand {\tup}[1] {\textup{#1}}
%
%\newcommand {\mbf}[1] {\mathbf{#1}}
%\newcommand {\mmd}[1] {\mathmd{#1}}
%\newcommand {\mrm}[1] {\mathrm{#1}}
%\newcommand {\msc}[1] {\mathsc{#1}}
%\newcommand {\msf}[1] {\mathsf{#1}}
%\newcommand {\msl}[1] {\mathsl{#1}}
%\newcommand {\mtt}[1] {\mathtt{#1}}
%\newcommand {\mup}[1] {\mathup{#1}}

%some math functions and symbols
\newcommand{\reals}{{\mathbb R}}
\newcommand{\expect}{{\mathbb E}}
\newcommand{\indicator}{{\mathbf 1}}
\newcommand{\ints}{{\mathbb Z}}
\newcommand{\complex}{{\mathbb C}}
\newcommand{\integers}{{\mathbb Z}}
\newcommand{\sign}[1]{\mathtt{sign}\left( #1 \right)}
\newcommand{\NP}{\operatorname{NPt}}
\newcommand{\erf}{\operatorname{erf}}
\newcommand{\NS}{\operatorname{NearestSet}}
\newcommand{\bres}{\operatorname{Bres}}
\newcommand{\vol}{\operatorname{vol}}
\newcommand{\vor}{\operatorname{Vor}}
%\newcommand{\re}{\operatorname{Re}}
%\newcommand{\im}{\operatorname{Im}}




\newcommand{\term}{\emph}
\newcommand{\var}{\operatorname{var}}
\newcommand{\covar}{\operatorname{cov}}
%\newcommand{\prob}{{\mathbb P}}
\newcommand{\prob}{{\operatorname{Pr}}}

%distribution fucntions
\newcommand{\projnorm}{\operatorname{ProjectedNormal}}
\newcommand{\vonmises}{\operatorname{VonMises}}
\newcommand{\wrapnorm}{\operatorname{WrappedNormal}}
\newcommand{\wrapunif}{\operatorname{WrappedUniform}}

\newcommand{\selectindicies}{\operatorname{selectindices}}
\newcommand{\sortindicies}{\operatorname{sortindices}}
\newcommand{\largest}{\operatorname{largest}}
\newcommand{\quickpartition}{\operatorname{quickpartition}}
\newcommand{\quickpartitiontwo}{\operatorname{quickpartition2}}

%some commonly used underlined and
%hated symbols
\newcommand{\uY}{\ushort{\mbf{Y}}}
\newcommand{\ueY}{\ushort{Y}}
\newcommand{\uy}{\ushort{\mbf{y}}}
\newcommand{\uey}{\ushort{y}}
\newcommand{\ux}{\ushort{\mbf{x}}}
\newcommand{\uex}{\ushort{x}}
\newcommand{\uhx}{\ushort{\mbf{\hat{x}}}}
\newcommand{\uehx}{\ushort{\hat{x}}}

% Brackets
\newcommand{\br}[1]{{\left( #1 \right)}}
\newcommand{\sqbr}[1]{{\left[ #1 \right]}}
\newcommand{\cubr}[1]{{\left\{ #1 \right\}}}
\newcommand{\abr}[1]{\left< #1 \right>}
\newcommand{\abs}[1]{\left\vert #1 \right\vert}
\newcommand{\sabs}[1]{\vert #1 \vert}
\newcommand{\babs}[1]{\big\vert #1 \big\vert}
\newcommand{\sfloor}[1]{{\lfloor #1 \rfloor}}
\newcommand{\floor}[1]{{\left\lfloor #1 \right\rfloor}}
\newcommand{\ceiling}[1]{{\left\lceil #1 \right\rceil}}
\newcommand{\ceil}[1]{\left\lceil #1 \right\rceil}
\newcommand{\round}[1]{{\left\lfloor #1 \right\rceil}}
\newcommand{\magn}[1]{\left\| #1 \right\|}
\newcommand{\fracpart}[1]{\left\langle #1 \right\rangle}
\newcommand{\sfracpart}[1]{\langle #1 \rangle}


% Referencing
\newcommand{\refeqn}[1]{\eqref{#1}}
\newcommand{\reffig}[1]{Figure~\ref{#1}}
\newcommand{\reftable}[1]{Table~\ref{#1}}
\newcommand{\refsec}[1]{Section~\ref{#1}}
\newcommand{\refappendix}[1]{Appendix~\ref{#1}}
\newcommand{\refchapter}[1]{Chapter~\ref{#1}}

\newcommand {\figwidth} {100mm}
\newcommand {\Ref}[1] {Reference~\cite{#1}}
\newcommand {\Sec}[1] {Section~\ref{#1}}
\newcommand {\App}[1] {Appendix~\ref{#1}}
\newcommand {\Chap}[1] {Chapter~\ref{#1}}
\newcommand {\Lem}[1] {Lemma~\ref{#1}}
\newcommand {\Thm}[1] {Theorem~\ref{#1}}
\newcommand {\Cor}[1] {Corollary~\ref{#1}}
\newcommand {\Alg}[1] {Algorithm~\ref{#1}}
\newcommand {\etal} {\emph{~et~al.}}
\newcommand {\bul} {$\bullet$ }   % bullet
\newcommand {\fig}[1] {Figure~\ref{#1}}   % references Figure x
\newcommand {\imp} {$\Rightarrow$}   % implication symbol (default)
\newcommand {\impt} {$\Rightarrow$}   % implication symbol (text mode)
\newcommand {\impm} {\Rightarrow}   % implication symbol (math mode)
\newcommand {\vect}[1] {\mathbf{#1}} 
\newcommand {\hvect}[1] {\hat{\mathbf{#1}}}
\newcommand {\del} {\partial}
\newcommand {\eqn}[1] {Equation~(\ref{#1})} 
\newcommand {\tab}[1] {Table~\ref{#1}} % references Table x
\newcommand {\half} {\frac{1}{2}} 
\newcommand {\ten}[1] {\times10^{#1}}
\newcommand {\bra}[2] {\mbox{}_{#2}\langle #1 |}
\newcommand {\ket}[2] {| #1 \rangle_{#2}}
\newcommand {\Bra}[2] {\mbox{}_{#2}\left.\left\langle #1 \right.\right|}
\newcommand {\Ket}[2] {\left.\left| #1 \right.\right\rangle_{#2}}
\newcommand {\im} {\mathrm{Im}}
\newcommand {\re} {\mathrm{Re}}
\newcommand {\braket}[4] {\mbox{}_{#3}\langle #1 | #2 \rangle_{#4}} 
\newcommand{\dotprod}[2]{ \left\langle #1 , #2 \right\rangle}
\newcommand {\trace}[1] {\text{tr}\left(#1\right)}

% spell things correctly
\newenvironment{centre}{\begin{center}}{\end{center}}
\newenvironment{itemise}{\begin{itemize}}{\end{itemize}}

%%%%% set up the bibliography style
\bibliographystyle{IEEEbib}
%\bibliographystyle{uqthesis}  % uqthesis bibliography style file, made
			      % with makebst

%%%%% optional packages
\usepackage[square,comma,numbers,sort&compress]{natbib}
		% this is the natural sciences bibliography citation
		% style package.  The options here give citations in
		% the text as numbers in square brackets, separated by
		% commas, citations sorted and consecutive citations
		% compressed 
		% output example: [1,4,12-15]

%\usepackage{cite}		
			
\usepackage{units}
	%nice looking units
		
\usepackage{booktabs}
		%creates nice looking tables
		
\usepackage{ifpdf}
\ifpdf
  \usepackage[pdftex]{graphicx}
  %\usepackage{thumbpdf}
  %\usepackage[naturalnames]{hyperref}
\else
	\usepackage{graphicx}% standard graphics package for inclusion of
		      % images and eps files into LaTeX document
\fi

\usepackage{amsmath,amsfonts,amssymb, amsthm, bm} % this is handy for mathematicians and physicists
			      % see http://www.ams.org/tex/amslatex.html

		 
\usepackage[vlined, linesnumbered]{algorithm2e}
	%algorithm writing package
	
\usepackage{mathrsfs}
%fancy math script

%\usepackage{ushort}
%enable good underlining in math mode

%------------------------------------------------------------
% Theorem like environments
%
\newtheorem{theorem}{Theorem}
%\theoremstyle{plain}
\newtheorem{acknowledgement}{Acknowledgement}
%\newtheorem{algorithm}{Algorithm}
\newtheorem{axiom}{Axiom}
\newtheorem{case}{Case}
\newtheorem{claim}{Claim}
\newtheorem{conclusion}{Conclusion}
\newtheorem{condition}{Condition}
\newtheorem{conjecture}{Conjecture}
\newtheorem{corollary}{Corollary}
\newtheorem{criterion}{Criterion}
\newtheorem{definition}{Definition}
\newtheorem{example}{Example}
\newtheorem{exercise}{Exercise}
\newtheorem{lemma}{Lemma}
\newtheorem{notation}{Notation}
\newtheorem{problem}{Problem}
\newtheorem{proposition}{Proposition}
\newtheorem{property}{Property}
\newtheorem{remark}{Remark}
\newtheorem{solution}{Solution}
\newtheorem{summary}{Summary}
%\numberwithin{equation}{section}
%--------------------------------------------------------


\usepackage{zref-xr,zref-user}
\zexternaldocument*{../paper}

\usepackage{tikz}
\usetikzlibrary{calc}
\usepackage{pgfplots}
\pgfplotsset{compat=1.8}

\usepackage[left=2cm,right=2cm,top=2cm,bottom=2cm]{geometry}

\usepackage{amsmath,amsfonts,amssymb, amsthm, bm}

%\usepackage[square,comma,numbers,sort&compress]{natbib}

\usepackage{color}
\newcommand{\comment}[1]{\textcolor{red}{#1}}

\newcommand{\sgn}{\operatorname{sgn}}
\newcommand{\sinc}{\operatorname{sinc}}
\newcommand{\rect}[1]{\operatorname{rect}\left(#1\right)}

%opening
\title{Fast sparse period estimation: Reviewer responses}
\author{R.~G.~McKilliam, I.~V.~L.~Clarkson and B.~G.~Quinn 
%\thanks{}
}

\renewcommand{\theenumii}{(\alph{enumii})}
\renewcommand{\labelenumii}{\theenumii}

\begin{document}

\maketitle


\section*{Reviewer 1}\label{sec:reviewer-1}

\begin{enumerate}
 
\item\textbf{Comment:}
This paper proposes a new fast method for estimating the period of sparse and noisy
observations of a point process. The new idea is to use the periodogram of quantized
observations that can be computed faster than the standard periodogram by using
a chirp z-transform or a fast Fourier transform. The reviewer is not convinced that
there is enough material deserving publication in the IEEE signal processing letters.
Moreover, there are several important comments that should be addressed before a
possible publication of this letter.
\\
\textbf{Response:}
We agree with the reviewer that the results in this paper are simple.  We do not believe this to be a weakness. Whilst being simple, the results are also novel and useful.  The quantized periodogram estimator described in this paper is as accurate as the most accurate estimators in the literature (the least squares and periodogram estimators).  At the same time our estimator is substantially faster to compute.  For these reasons, we firmly believe this will become the estimator of choice for periodic point processes.  It enables highly accurate period estimation for extremely large data sets in reasonable time. This was not possible with previous estimators that require a number of operations growing quadratically with data size. %We have provided all the simulation code as supplementary material.  This will enable simple implementation for future research.

\item\textbf{Comment:}
The main reviewer concern is about the choice of the quantization parameter
$q$ that should be explained more carefully. It is not sufficient to display two sets
of curves obtained for two different values of $q$ and to choose the best one without
explanations. For instance, it is not clear whether the value of $q$ chosen in this paper
will be appropriate for other noise distributions or for other ways of generating the
integers $s_n$. This point should be addressed carefully before a possible resubmission
of the paper. The other reviewer comments are summarized below.
\\
\textbf{Response:}
We anticipate that developing a rigorous theoretical method for choosing $q$ will be extremely complicated.  The properties of many optimisation procedures, such as the Newton-Raphson method, are reasonably well understood for the purpose of maximising fixed known functions. However, very few theoretical statistical analyses for these procedures exist when the functions to be maximised are themselves random variables, i.e., when noise is involved.  Add to this the existence of quantization (itself typically very tricky to analyse) and it appears that a theoretical analysis of this estimator will be extremely difficult.

These theoretical difficulties do not negate the immediate practical applicability of our estimator.  We believe it would be unwise to halt dissemination of these practical results simply because a difficult theoretical problem has not yet been solved.  We agree with the reviewer that the simulations presented are limited. This is due to the 4 page limit required IEEE Signal Processing Letters (a 5th page containing only references is also allowed).  In the paper we use our simulations to motivate the choice $q = 5 f_{\text{max}}$ for the quantization parameter.  We have found this choice works well under an array of different noise distributions and methods for generating the integers $s_1,\dots,s_N$.  For example, Figures~\ref{plot:poisson} and~\ref{plot:geomuniform} in this document show simulations similar to those in the paper, but with the following modifications:

\begin{enumerate}
\item In Figure~\ref{plot:poisson} the noise is Gaussian distributed, $N=200$, and the integers are generated so that $s_1$ and $s_{n+1} - s_n$ for $n=1,\dots,N-1$ are independent and identically \emph{Poisson distributed} with mean $\mu = 1$ in the plot on the left and mean $\mu=10$ in the plot on the right.  Five hundred Monte-Carlo trials are used for each value of $\sigma^2$.
\item In Figure~\ref{plot:geomuniform} the noise if \emph{uniformly distributed}, $N=200$, and the integers are generated so that $s_1$ and $s_{n+1} - s_n$ for $n=1,\dots,N-1$ are independent and identically geometrically distributed with mean $\mu = 1$ in the plot on the left and mean $\mu=10$ in the plot on the right.  Five hundred Monte-Carlo trials are used for each value of $\sigma^2$.   The least squares estimator is the most accurate in this simulation. This is predicted by the asymptotic theory~\cite{Quinn_sparse_noisy_SSP_2012,Quinn20013asilomar_period_est}.
\end{enumerate}

Observe that, in both simulations, the choice $q=5 f_{\text{max}}$ provides good accuracy. With this choice, the accuracy of the quantized periodogram estimator is similar to the original periodogram estimator computed without quantization and FFT.  %That the choice $q=5 f_{\text{max}}$ works in a wide range of scenarios should not come as a  


\begin{figure}[p] 
  \centering 
  \begin{tikzpicture} 
    \selectcolormodel{gray} 
    \begin{axis}[name=plot1,font=\footnotesize,xmode=log,ymax=4e-1,ymode=log,height=10cm,width=9cm,xlabel={$\sigma^2$},ylabel={MSE},ylabel style={at={(0.06,0.24)}},xlabel style={at={(0.5,0.07)}}, legend style={at={(1.1,1.02)}, anchor=south,legend columns=7, cells={anchor=west,/tikz/every even column/.append style={column sep=0.2cm}}}]
      \addplot[mark=none,color=black] table {code/data/PeriodogramCLTN200poisson1}; 
      \addplot[mark=none,color=black,dashed] table {code/data/NormalisedLLSCLTN200poisson1};
            \addplot[mark=triangle,only marks,mark options={solid,fill=black,scale=1}] table {code/data/QuantisedPeriodogramFFTN200q1.5poisson1};
      \addplot[mark=*,only marks,color=black,mark options={solid,fill=black,scale=0.5}] table {code/data/QuantisedPeriodogramFFTN200q5.0poisson1};
      \addplot[mark=o,only marks,color=black,mark options={solid,fill=black,scale=1}] table {code/data/PeriodogramN200poisson1};
       \addplot[mark=x,only marks,color=black,mark options={solid,fill=black,scale=1.2}] table {code/data/NormalisedSamplingLLSN200poisson1}; 
 %      \addplot[mark=square,only marks,color=black,mark options={solid,fill=black,scale=0.8}] table {code/data/SLS2novlpN200poisson1};
      \addplot[mark=square,only marks,color=black,mark options={solid,fill=black,scale=1}] table {code/data/SLS2newN200poisson1};
     \legend{Periodogram theory, LS theory, $q=\tfrac{3}{2}f_{\text{max}}$, $q=5 f_{\text{max}}$, Periodogram, LS, SLS2-new}
     %\node[rotate=17,font=\footnotesize] at (axis cs:6e-4,1e-7) {$N=200$};
   \end{axis} 
 \begin{axis}[name=plot2,at={(10.5,0)},ymax=4e-1,font=\footnotesize,xmode=log,ymode=log,height=10cm,width=9cm,xlabel={$\sigma^2$},ylabel={MSE},ylabel style={at={(0.06,0.23)}},xlabel style={at={(0.5,0.07)}}]
      \addplot[mark=none,color=black] table {code/data/PeriodogramCLTN200poisson10}; 
      \addplot[mark=none,color=black,dashed] table {code/data/NormalisedLLSCLTN200poisson10};
            \addplot[mark=triangle,only marks,mark options={solid,fill=black,scale=1}] table {code/data/QuantisedPeriodogramFFTN200q1.5poisson10};
      \addplot[mark=*,only marks,color=black,mark options={solid,fill=black,scale=0.5}] table {code/data/QuantisedPeriodogramFFTN200q5.0poisson10};
      \addplot[mark=o,only marks,color=black,mark options={solid,fill=black,scale=1.1}] table {code/data/PeriodogramN200poisson10};
       \addplot[mark=x,only marks,color=black,mark options={solid,fill=black,scale=1.3}] table {code/data/NormalisedSamplingLLSN200poisson10}; 
  %     \addplot[mark=square,only marks,color=black,mark options={solid,fill=black,scale=0.8}] table {code/data/SLS2novlpN200poisson10};
       \addplot[mark=square,only marks,color=black,mark options={solid,fill=black,scale=1}] table {code/data/SLS2newN200poisson10};
     %\node[rotate=16,font=\footnotesize] at (axis cs:7e-4,1.5e-14) {$N=200$}; 
   \end{axis} 
  \end{tikzpicture}  
  \caption{Mean square period error versus noise variance $\sigma^2$ with Gaussian noise and $N=200$ observations.  The integers $s_1,\dots,s_N$ are generated so that $s_1$ and $s_{n+1} - s_n$ for $n=1,\dots,N-1$ are independent and identically Poisson distributed with mean $\mu = 1$ in the plot on the left and mean $\mu=10$.}\label{plot:poisson}
\end{figure} 


\begin{figure}[p]  
  \centering
  \begin{tikzpicture}  
    \selectcolormodel{gray} 
    \begin{axis}[name=plot1,font=\footnotesize,xmode=log,ymax=4e-1,ymode=log,height=10cm,width=9cm,xlabel={$\sigma^2$},ylabel={MSE},ylabel style={at={(0.06,0.24)}},xlabel style={at={(0.5,0.07)}}, legend style={at={(1.1,1.02)}, anchor=south,legend columns=7, cells={anchor=west,/tikz/every even column/.append style={column sep=0.2cm}}}]
\addplot[mark=none,color=black] table {code/data/PeriodogramCLTN200uniformgeom1}; 
      \addplot[mark=none,color=black,dashed] table {code/data/NormalisedLLSCLTN200uniformgeom1};
            \addplot[mark=triangle,only marks,mark options={solid,fill=black,scale=1}] table {code/data/QuantisedPeriodogramFFTN200q1.5uniformgeom1};
      \addplot[mark=*,only marks,color=black,mark options={solid,fill=black,scale=0.5}] table {code/data/QuantisedPeriodogramFFTN200q5.0uniformgeom1};
      \addplot[mark=o,only marks,color=black,mark options={solid,fill=black,scale=1}] table {code/data/PeriodogramN200uniformgeom1};
       \addplot[mark=x,only marks,color=black,mark options={solid,fill=black,scale=1.2}] table {code/data/NormalisedSamplingLLSN200uniformgeom1}; 
 %      \addplot[mark=square,only marks,color=black,mark options={solid,fill=black,scale=0.8}] table {code/data/SLS2novlpN200uniformgeom1};
      \addplot[mark=square,only marks,color=black,mark options={solid,fill=black,scale=1}] table {code/data/SLS2newN200uniformgeom1};
     \legend{Periodogram theory, LS theory, $q=\tfrac{3}{2}f_{\text{max}}$, $q=5 f_{\text{max}}$, Periodogram, LS, SLS2-new}
     %\node[rotate=17,font=\footnotesize] at (axis cs:6e-4,1e-7) {$N=200$};
   \end{axis}   \begin{axis}[name=plot2,at={(4.7,0)},ymax=4e-1,font=\footnotesize,xmode=log,ymode=log,height=10cm,width=9cm,xlabel={$\sigma^2$},ylabel={MSE},ylabel style={at={(0.06,0.23)}},xlabel style={at={(0.5,0.07)}}]
\addplot[mark=none,color=black] table {code/data/PeriodogramCLTN200uniformgeom10}; 
      \addplot[mark=none,color=black,dashed] table {code/data/NormalisedLLSCLTN200uniformgeom10};
            \addplot[mark=triangle,only marks,mark options={solid,fill=black,scale=1}] table {code/data/QuantisedPeriodogramFFTN200q1.5uniformgeom10};
      \addplot[mark=*,only marks,color=black,mark options={solid,fill=black,scale=0.5}] table {code/data/QuantisedPeriodogramFFTN200q5.0uniformgeom10};
      \addplot[mark=o,only marks,color=black,mark options={solid,fill=black,scale=1.1}] table {code/data/PeriodogramN200uniformgeom10};
       \addplot[mark=x,only marks,color=black,mark options={solid,fill=black,scale=1.3}] table {code/data/NormalisedSamplingLLSN200uniformgeom10}; 
  %     \addplot[mark=square,only marks,color=black,mark options={solid,fill=black,scale=0.8}] table {code/data/SLS2novlpN200uniformgeom10};
       \addplot[mark=square,only marks,color=black,mark options={solid,fill=black,scale=1}] table {code/data/SLS2newN200uniformgeom10};
     %\node[rotate=16,font=\footnotesize] at (axis cs:7e-4,1.5e-14) {$N=200$}; 
   \end{axis} 
  \end{tikzpicture}  
  \caption{Mean square period error versus noise variance $\sigma^2$ with uniform noise and $N=200$ observations.  The integers $s_1,\dots,s_N$ are generated so that $s_1$ and $s_{n+1} - s_n$ for $n=1,\dots,N-1$ are independent and identically geometrically distributed with mean $\mu = 1$ in the plot on the left and mean $\mu=10$.  The least squares estimator is the most accurate in this simulation. This is predicted by the asymptotic theory~\cite{Quinn_sparse_noisy_SSP_2012,Quinn20013asilomar_period_est}.}\label{plot:geomuniform}
\end{figure}


\item\textbf{Comment:}
p. 3, the authors mention “Sadler and Casey [24] \dots However, their estimator
requires far larger transforms and produces less accurate results than the estimator
described here (see Section VI)". Unfortunately, there is nothing in
Section VI (which is the conclusion) supporting this claim. The authors should
compare their results in term of accuracy and of computational complexity to
the results of [24]. Including the Cramer-Rao bounds for the estimation of $T_0$
(as in [24]) would also be appreciated.
\\
\textbf{Response:}

We respond to each of the reviewers comments in turn.

\begin{enumerate}

\item \textbf{Unfortunately, there is nothing in
Section VI (which is the conclusion) supporting this claim.}

Section VI in the paper~\cite{726812} entitled ``On periodic pulse interval analysis with outliers and missing observations'' is not the conclusion.  It is the section entitled ``Numerical Results''.  In Figures 1 to 4 in~\cite{726812} the periodogram estimator implemented by Casey and Sadler is seen to be inferior to their MEA estimator.  Casey and Sadler do use an FFT to compute the periodogram, but fail to implement an optimisation procedure. We used the Newton-Raphson proceedure (see our response to comment~\ref{com:rev1:nr} below).  Casey and Sadler arrive at the conclusion that very large FFTs are required and that the periodogram estimator is not particularly accurate.  In the conclusion of~\cite{726812} it is stated that:

``Comparisons with the periodogram show the limitations of periodogram performance.''

Our paper clearly shows that such limitation do not exist.  The periodogram estimator, computed by quantization and FFT and combined with an appropriate optimisation procedure, is both computationally efficient and highly accurate.  This is the most computationally efficient procedure currently available for this problem.  Simultaneously, it is as accurate at the most accurate estimators available, namely the least squares estimator, and the periodogram estimator computed directly without FFT.

\item \textbf{ The authors should
compare their results in term of accuracy and of computational complexity to
the results of [24].}

We have not presented simulation results with the MEA estimators of Casey and Sadler~\cite{536682,726812}.  The MEA estimators were superseded by the Separable Line Search (SLS) estimators of Sidiropoulos, Swami and Sadler~\cite{Sidiropoulos2005} in 2005.  Comparison between the MEA estimators and the SLS estimators are presented in~\cite{Sidiropoulos2005} where the superiority of the SLS estimators is shown.  An enhanced SLS estimator (SLS-new) was developed in~\cite{Clarkson2007} and this is the estimator we show simulations results for\footnote{One of the SLS estimators (SLS2-all) has excellent statistical accuracy~\cite{Sidiropoulos2005}.  It is almost as accurate as the least squares and periodogram estimators as can be see in the simulations in~\cite{Clarkson2007} and~\cite{McKilliam2007}.  However, it is extremely computationally complex, requiring at least $O(N^3)$ operations.  We cannot present simulation results for this estimator when $N=1200$ for example.  The SLS2-all estimator is discussed and compared in~\cite{Sidiropoulos2005} and also~\cite{Clarkson2007,McKilliam2007}.}.  Even this enhanced SLS estimator performs poorly when compared to the least squares and periodogram estimators.  The MEA estimators are no longer state-of-the-art for this problem.

\item \textbf{Including the Cramer-Rao bounds for the estimation of $T_0$
(as in [24]) would also be appreciated.}

In the simulations we regenerate the integers $s_1,\dots,s_N$ on each Monte-Carlo trial.  To make this clear in the paper the following sentence is included:

``For each Monte-Carlo trial the integers $s_1,\dots,s_N$ are generated so that $s_1$ and $s_{n+1} - s_n$ for $n=1,\dots,N-1$ are independent and identically geometrically distributed starting from zero with means $\mu=1$ and $10$.''
 
To the authors knowledge, a Cramer-Rao bound is not known in this case.   The \emph{clairvoyant} Cramer-Rao bound from~\cite{Sidiropoulos2005} (also \cite{726812,Clarkson2007}) supposes a single fixed realisation for $s_1,\dots,s_N$.  This bound is therefore not valid for the simulations we present in Figure~\zref{plot:multipleN} in Section~\zref{sec:simulations} of the paper.  

In Figure~\ref{plot:crb} in this document we perform a simulation with fixed integers $s_1,\dots,s_N$. The integers are generated once such $s_1$ and $s_{n+1} - s_n$ for $n=1,\dots,N-1$ are independent and identically geometrically distributed.  This single fixed sequence is used for all 500 Monte-Carlo trials.  Observe that for small noise variance the \emph{clairvoyant} Cramer-Rao bound is similar to the asymptotic variance of the least squares and periodogram estimators predicted in~\cite{Quinn_sparse_noisy_SSP_2012,Quinn20013asilomar_period_est}.  As the noise variance increases the \emph{clairvoyant} Cramer-Rao bound and the predicted asymptotic variances diverge.  As expected, the behaviour of the estimators measured by simulation follows the asymptotic theory, rather than the \emph{clairvoyant} Cramer-Rao bound.

Observe that in the left hand plot of Figure~\ref{plot:crb} the quantized periodogram estimator with coarse quantization $q = 1.5 f_{\text{max}}$ (triangles) actually performs well for sufficiently small noise variance.  It appears that the fixed integers $s_1,\dots,s_N$ generated in this simulation favour the quantized periodogram estimator.  This is one motivation for performing simulations with integers $s_1,\dots,s_N$ generated randomly for each Monte-Carlo trial.  It avoids the scenario where a particular favourable set $s_1,\dots,s_N$ is chosen by chance.


%We prefer to show to plot the MSE of the estimatorsThe least squares estimator is the maximum likelihood estimator, and so, the dashed curve serves a similar type of purpose as the CRB often does in these types of simulations.  Namely, to indicate whether an estimator is performing similarly to what might be considered `optimal', i.e. close the what might theoretically be expected from the maximum likelihood estimator.

\end{enumerate}

\item\textbf{Comment:}\label{com:rev1:nr}
p. 7, the authors indicate that they have used the Newton-Raphson (NR)
method. More details should be provided here. For instance, what is the exact
NR scheme used in this paper (there are several options)? How have the
authors selected the total number of iterations? Is the quantized periodogram
just used to initialize a Newton-Raphson procedure? Do the MSEs displayed
in Fig. 3 correspond to the estimator obtained after applying the NR algorithm?
Also, do the results displayed in Fig. 2 correspond to the total time
required to compute the estimator (including the NR procedure) or just to the
initialization of the NR procedure?
\\
\textbf{Response:}
We respond to each of the questions of the reviewer in turn.

\begin{enumerate}
\item \textbf{For instance, what is the exact NR scheme used in this paper (there are several options)?}  
In the new version of the paper the following paragraph has been included in Section~\zref{sec:comp-compl}:

``Given the approximate maximiser $\widetilde{f}^\prime$ we apply the Newton-Raphson method~\cite{Ypma_historical_newton_raphson_1995} to find a nearby stationary point of the periodogram $I_y$.  This proceeds by the iteration
\[
f_{n+1} = f_{n} - \frac{I_y^\prime(f_n)}{I_y^{\prime\prime}(f_n)}, \qquad f_{1} = \widetilde{f}^\prime
\]
where $I_y^\prime$ and $I_y^{\prime\prime}$ are the first and second derivative of $I_y$. This procedure converges to the maximiser $\hat{f}$ of $I_y$ provided that $\widetilde{f}^\prime$ is sufficiently close to $\hat{f}$.  In our implementation iterations are performed until $\sabs{f_{n+1}-f_n} < 10^{-10}$.  The number of iterations depends upon the distance $D$, but not on $N$.  Each iteration requires $O(N)$ operations and so the total number of operations required by the Newton-Raphson procedure has order $O(N)$.  In practice we find that the computational cost of the Newton-Raphson procedure is negligible when compared to the cost of the chirp-z or fast Fourier transforms.''

% The Netwon-Raphson method that was used proceeds by the iteration
% \[
% x_{n+1} = x_n - \frac{f^\prime(x_{n})}{f^{\prime\prime}(x_n)}
% \]
% to find a stationary point of the function $f$ where $f^\prime$ and $f^{\prime\prime}$ are the first and second derivatives of $f$.  To avoid the potential ambguity regarding the definition of the Newton-Raphson method we have, in the new manuscipt, replaced the Newton-Raphson method with the optimisation proceedure due to Brent~\cite[Ch.~5]{Brent_opt_no_derivs_1973}.  This optimiation method is likely to be more familiar with readers because it is used internally by the \texttt{fminbnd} function in the popular MATLAB environment. 

% BLERG add some text to the paper.

\item \textbf{How have the authors selected the total number of iterations?} 
As described in the above paragraph ``iterations are performed until $\sabs{f_{n+1}-f_n} < 10^{-10}$.''

% In our implementation iterations are performed until the absolute difference between consequentive iterations is less than $10^{-7}$.

% BLERG: Something like the following should be added:

% ``Brent's method is iterative.  During the method a number of evaluations of the periodogram $I_y$ are made.  The number of calls depends on the grid width $D$, but not on $N$.  Each call requires $O(N)$ operations, and so, the total number of operations required by Brent's method is $O(N)$. In practice we find that that computational cost of Brent's method is far smaller than the cost of either that FFT or chirp-z transform.''

\item \textbf{Is the quantized periodogram just used to initialize a Newton-Raphson procedure?}
Yes.  In the paper the following text is included in Section~\zref{sec:quant-peri}:

``Let $\widetilde{f}^\prime$ be the maximiser of $I_z$ over the grid $G$, that is,
\[
\widetilde{f}^\prime = \arg\max_{f \in G} I_z(f).
\]
If the quantization parameter $q$ is sufficiently large we expect that $I_z$ is close to $I_y$ and so $\widetilde{f}^\prime$ will be close to $\widetilde{f}$.  Thus, if $q$ is sufficiently large and the distance $D$ is sufficiently small we expect $\widetilde{f}^\prime$ to be a good approximation to $\hat{f}$.  An iterative procedure %or Brent's method~\cite{Brent_opt_no_derivs_1973} 
can then be used to compute $\hat{f}$ by maximising $I_y$ starting from $\widetilde{f}^\prime$.  Section~\zref{sec:comp-compl} describes an implementation of the Newton-Raphson optimisation procedure for this purpose.''

\item \textbf{Do the MSEs displayed in Fig. 3 correspond to the estimator obtained after applying the NR algorithm?}
Yes.

\item \textbf{Also, do the results displayed in Fig. 2 correspond to the total time
required to compute the estimator (including the NR procedure) or just to the
initialization of the NR procedure?}
The estimators run in the benchmarks in Figure~\zref{plot:benchmark} are exactly the same as those in the simulations in Figure~\zref{plot:multipleN}.  For the periodogram and quantized periodogram estimators this includes the Newton-Raphson procedure.

\end{enumerate}



\item\textbf{Comment:}
p. 5, in the definition of $v_m$, $b_{m+\ell_1}$ should be replaced by $b_{m+\ell_{\text{min}}}$.
\\
\textbf{Response:}
Thank you for spotting this error.  A similar error was fixed in equation for $I_z$ directly above this. 


\item\textbf{Comment:}
p. 6, ``each of length $L+K-1$'' might be replaced by ``each of length $M = L + K - 1$"
\\
\textbf{Response:}
Respectfully, we have not done this.  The parameter $M$ is reserved for the length of the FFT used for the single Fourier transform method.  We feel it is confusing if $M$ is also used for the length of the transforms required by the chirp-z transform.


\item\textbf{Comment:}
p. 8, $n = 2,\dots,N$ should be $n = 1, \dots, N - 1$
\\
\textbf{Response:}
This error has be fixed.  The same error in the caption of Figure~\zref{plot:multipleN} has been fixed.


\item\textbf{Comment:}
p. 8, it is not completely clear to identify the curves associated with $N = 30$ and $N = 1200$. Can the authors display 4 curves associated with $\mu= 1$, $\mu = 10$, $N = 30$ and $N = 1200$?
\\
\textbf{Response:}
The formatting of Figure~\zref{plot:multipleN} has been improved.  Each simulation for $N=30,1200$ and $\mu=1,10$ is now presented in its own plot.

\item\textbf{Comment:}
Typos: p. 2, ``anomolies'' should be ``anomalies'', p. 7, ``quanitised'' should be
``quantized''.
\\
\textbf{Response:}
Done.


\item\textbf{Comment:} 
Is it necessary to include 11 self citations in this letter? Removing some of
these references would allow to save some space, which could be used to better
explain some points mentioned above.
\\
\textbf{Response:}
IEEE Signal Processing Letters allows a 5th page that can contain only references.  Therefore, we cannot save space by removing references.   We have removed reference [29] from the original manuscript because the material in that paper is also covered in [27].  We have also removed references [33] and [34] from the original manuscript. We feel that all 35 references in the new version of the paper are highly relevant. The paper would be weakened without them.

\end{enumerate}



\begin{figure}[tp]  
  \centering
  \begin{tikzpicture}  
    \selectcolormodel{gray}
    \begin{axis}[name=plot1,font=\footnotesize,xmode=log,ymax=4e-1,ymode=log,height=10cm,width=9cm,xlabel={$\sigma^2$},ylabel={MSE},ylabel style={at={(0.06,0.24)}},xlabel style={at={(0.5,0.07)}}, legend style={at={(1.1,1.02)}, anchor=south,legend columns=8, cells={anchor=west,/tikz/every even column/.append style={column sep=0.2cm}}}]
\addplot[mark=none,color=black] table {code/data/PeriodogramCLTN500fixed1}; 
      \addplot[mark=none,color=black,dashed] table {code/data/NormalisedLLSCLTN500fixed1};
      \addplot[mark=none,color=black,dotted] table {code/data/CRBN500fixed1};
            \addplot[mark=triangle,only marks,mark options={solid,fill=black,scale=1}] table {code/data/QuantisedPeriodogramFFTN500q1.5fixed1};
      \addplot[mark=*,only marks,color=black,mark options={solid,fill=black,scale=0.5}] table {code/data/QuantisedPeriodogramFFTN500q5.0fixed1};
      \addplot[mark=o,only marks,color=black,mark options={solid,fill=black,scale=1}] table {code/data/PeriodogramN500fixed1};
       \addplot[mark=x,only marks,color=black,mark options={solid,fill=black,scale=1.2}] table {code/data/NormalisedSamplingLLSN500fixed1}; 
 %      \addplot[mark=square,only marks,color=black,mark options={solid,fill=black,scale=0.8}] table {code/data/SLS2novlpN500fixed1};
      \addplot[mark=square,only marks,color=black,mark options={solid,fill=black,scale=1}] table {code/data/SLS2newN500fixed1};
     \legend{Periodogram theory, LS theory, CRB, $q=\tfrac{3}{2}f_{\text{max}}$, $q=5 f_{\text{max}}$, Periodogram, LS, SLS2-new}
     %\node[rotate=17,font=\footnotesize] at (axis cs:6e-4,1e-7) {$N=200$};
   \end{axis}    \begin{axis}[name=plot2,at={(9.5,0)},ymax=4e-1,font=\footnotesize,xmode=log,ymode=log,height=10cm,width=9cm,xlabel={$\sigma^2$},ylabel={MSE},ylabel style={at={(0.06,0.23)}},xlabel style={at={(0.5,0.07)}}]
\addplot[mark=none,color=black] table {code/data/PeriodogramCLTN500fixed10}; 
      \addplot[mark=none,color=black,dashed] table {code/data/NormalisedLLSCLTN500fixed10};
      \addplot[mark=none,color=black,dotted] table {code/data/CRBN500fixed10};
            \addplot[mark=triangle,only marks,mark options={solid,fill=black,scale=1}] table {code/data/QuantisedPeriodogramFFTN500q1.5fixed10};
      \addplot[mark=*,only marks,color=black,mark options={solid,fill=black,scale=0.5}] table {code/data/QuantisedPeriodogramFFTN500q5.0fixed10};
      \addplot[mark=o,only marks,color=black,mark options={solid,fill=black,scale=1.1}] table {code/data/PeriodogramN500fixed10};
       \addplot[mark=x,only marks,color=black,mark options={solid,fill=black,scale=1.3}] table {code/data/NormalisedSamplingLLSN500fixed10}; 
  %     \addplot[mark=square,only marks,color=black,mark options={solid,fill=black,scale=0.8}] table {code/data/SLS2novlpN500fixed10};
       \addplot[mark=square,only marks,color=black,mark options={solid,fill=black,scale=1}] table {code/data/SLS2newN500fixed10};
     %\node[rotate=16,font=\footnotesize] at (axis cs:7e-4,1.5e-14) {$N=200$}; 
   \end{axis} 
  \end{tikzpicture}
  \caption{Mean square period error versus noise variance $\sigma^2$ with Gaussian noise and $N=500$ observations.  A singled fixed set of integers $s_1,\dots,s_N$ are generated so that $s_1$ and $s_{n+1} - s_n$ for $n=1,\dots,N-1$ are independent and identically geometrically distributed with mean $\mu = 1$ in the plot on the left and mean $\mu=10$.  This single fixed sequence is used for all of the $500$ Monte-Carlo trials.  In this case the \emph{clairvoyant} Cramer-Rao bound from~\cite{Sidiropoulos2005} (also in \cite{726812,Clarkson2007}) exists and is displayed by the dotted line.  In this simulation the \emph{clairvoyant} Cramer-Rao bound is equal to the variance of the maximum likelihood (i.e. least squares) estimator in the case that the integers $s_1,\dots,s_N$ are known.}\label{plot:crb}
\end{figure} 


\section*{Reviewer 2}


\begin{enumerate}

\item\textbf{Comment:}
 The identifiability claim at the beginning of sec. II is not obvious and a reference is mandatory. Indeed, referring to eq. (2), let us assume that $s_n = 2 i_n$ where $i_n$ are all non-negative integer. In this case the point process is periodic with period $2 T_0$. This referee does not believe that maximizing the periodogram over $(f_{\text{min}},f_{\text{max}})$ will identify the true frequency (or period). LS methods might have the capabilities to identify the period, but, in any case the claim of idenitifiability needs to be further supported.
\\
\textbf{Response:} The same identifiability requirements are commonly imposed in the literature, see for example, Sidiropoulos~\cite[eq. (2)]{Sidiropoulos2005}.  To  see why this is necessary, suppose that $T_{\text{max}} \geq 2T_{\text{min}}$ and that the true period $T_0$ happened to be contained in the nonempty interval $(2T_{\text{min}}, T_{\text{max}})$.  Put $T^\prime = T_0/2 \in (T_{\text{min}}, T_{\text{max}})$ and observe that
\[
y_n = T_0 s_n + \theta_0 + w_n = 2 T^\prime s_n + \theta_0 + w_n = T^\prime r_n + \theta_0 + w_n 
\]
where $r_n = 2 s_n \in \ints$.  Because the integers $s_1,\dots,s_N$ (respectively $r_1,\dots,r_N$) are not known both $T_0$ and $T^\prime$ would describe the observations equally well.  This problem is averted only if we suppose that $T_0 \in (T_{\text{min}}, T_{\text{max}})$ and that  $T_{\text{max}} < 2T_{\text{min}}$.

%Perhaps it is possible to increase the range of identifiability if further assumptions are made regarding the integers $s_1,\dots,s_N$.  We don't consider such propsitions in this paper.

% To see this another way, define least squares objective function $LS : \reals \times \reals \times \ints^N \mapsto \reals$ as
% \[
% LS(T, \theta, z_1,\dots,z_N) = \sum_{n=1}^N (y_n - T z_n - \theta)^2.
% \]
% In the case that the noise variables $w_1,\dots,w_n$ are i.i.d. and normally distributed the minimiser of this objective function with respect to $T$ is the maximum likelihood estimator.  That is,
% \[
% \hat{T} = \arg\min_{T \in (T_{\text{min}}, T_{\text{max}})} \min_{\theta \in \reals} \min_{z_1,\dots,z_N \in \ints} LS(T, \theta, z_1,\dots,z_N)
% \]
% is the maximum likelihood period estimator.

We have included reference~\cite{Sidiropoulos2005} and simplified the text at the start of Section~\zref{sec:peri-estim} to read:

``To ensure identifiability it is necessary to assume, as in~\cite{Sidiropoulos2005}, that $T_0$ lies in some interval $(T_{\text{min}}, T_{\text{max}})$ where $T_{\text{max}} < 2T_{\text{min}}$.''

\item\textbf{Comment:}
Is the equality in the second (un-numbered) equation of page 2 correct?
\\
\textbf{Response:}
Yes.  The terms inside the sum on the left and right are conjugates.  That is,
\[
\sum_{n=1}^N e^{ 2\pi j f y_n} = \sum_{n=1}^N (e^{-2\pi j f y_n})^* = \left( \sum_{n=1}^N e^{ -2\pi j f y_n} \right)^*
\]
where $^*$ denotes the complex conjugate.  So,
\[
I_y(f) = \babs{ \sum_{n=1}^N e^{ 2\pi j f y_n} }^2 = \babs{ \sum_{n=1}^N e^{ -2\pi j f y_n} }^2.
\]

\item\textbf{Comment:}
Newton-Raphson iteration to further improve the frequency estimation is advocated in Sec. II. Can a reference supporting this statement be added?
\\
\textbf{Response:} 
Please see our response to comment~\ref{com:rev1:nr} of reviewer 1.  The reference~\cite{Ypma_historical_newton_raphson_1995} that gives an overview of the Newton-Raphson method has also been included.

\item\textbf{Comment:}
Referring to Fig. 3, in case $N=30$, the novel method incurs in a significant loss of performance with respect to classical LS and periodogram for low-signal-to-noise ratios. In this scenario (low $N$ and low signal-to-noise-ration) relevant for real applications?
\\
\textbf{Response:}
The quantized periodogram estimator does not suffer a significant performance loss with respect to the classical LS and periodogram estimators in any scenario we have found.  Observe in Figure 3 that the dots ($q=5 f_{\text{max}}$), the crosses (LS), and the circles (periodogram) are all close for $N=30,1200$ and $\mu=1,10$.  The solid and dashed lines indicate the asymptotic theory ($N\to\infty$)~\cite{Quinn_sparse_noisy_SSP_2012,Quinn20013asilomar_period_est} for the periodogram and LS estimators and not the measured MSE for finite $N$.  As $N\to\infty$, the MSE of the periodogram and LS estimators will converge to the solid and dashed curves.

In general we don't expect there to exist an estimator accurate for small $N$ and large noise variance.

\end{enumerate}



{\small
%\documentclass[a4paper,10pt]{article}
%\documentclass[draftcls, onecolumn, 11pt]{IEEEtran}
\documentclass[10pt,twocolumn,twoside]{IEEEtran}
%\documentclass[journal]{IEEEtran}
 
\usepackage{mathbf-abbrevs}
%\newcommand {\tbf}[1] {\textbf{#1}}
%\newcommand {\tit}[1] {\textit{#1}}
%\newcommand {\tmd}[1] {\textmd{#1}}
%\newcommand {\trm}[1] {\textrm{#1}}
%\newcommand {\tsc}[1] {\textsc{#1}}
%\newcommand {\tsf}[1] {\textsf{#1}}
%\newcommand {\tsl}[1] {\textsl{#1}}
%\newcommand {\ttt}[1] {\texttt{#1}}
%\newcommand {\tup}[1] {\textup{#1}}
%
%\newcommand {\mbf}[1] {\mathbf{#1}}
%\newcommand {\mmd}[1] {\mathmd{#1}}
%\newcommand {\mrm}[1] {\mathrm{#1}}
%\newcommand {\msc}[1] {\mathsc{#1}}
%\newcommand {\msf}[1] {\mathsf{#1}}
%\newcommand {\msl}[1] {\mathsl{#1}}
%\newcommand {\mtt}[1] {\mathtt{#1}}
%\newcommand {\mup}[1] {\mathup{#1}}

%some math functions and symbols
\newcommand{\reals}{{\mathbb R}}
\newcommand{\expect}{{\mathbb E}}
\newcommand{\indicator}{{\mathbf 1}}
\newcommand{\ints}{{\mathbb Z}}
\newcommand{\complex}{{\mathbb C}}
\newcommand{\integers}{{\mathbb Z}}
\newcommand{\sign}[1]{\mathtt{sign}\left( #1 \right)}
\newcommand{\NP}{\operatorname{NPt}}
\newcommand{\erf}{\operatorname{erf}}
\newcommand{\NS}{\operatorname{NearestSet}}
\newcommand{\bres}{\operatorname{Bres}}
\newcommand{\vol}{\operatorname{vol}}
\newcommand{\vor}{\operatorname{Vor}}
%\newcommand{\re}{\operatorname{Re}}
%\newcommand{\im}{\operatorname{Im}}




\newcommand{\term}{\emph}
\newcommand{\var}{\operatorname{var}}
\newcommand{\covar}{\operatorname{cov}}
%\newcommand{\prob}{{\mathbb P}}
\newcommand{\prob}{{\operatorname{Pr}}}

%distribution fucntions
\newcommand{\projnorm}{\operatorname{ProjectedNormal}}
\newcommand{\vonmises}{\operatorname{VonMises}}
\newcommand{\wrapnorm}{\operatorname{WrappedNormal}}
\newcommand{\wrapunif}{\operatorname{WrappedUniform}}

\newcommand{\selectindicies}{\operatorname{selectindices}}
\newcommand{\sortindicies}{\operatorname{sortindices}}
\newcommand{\largest}{\operatorname{largest}}
\newcommand{\quickpartition}{\operatorname{quickpartition}}
\newcommand{\quickpartitiontwo}{\operatorname{quickpartition2}}

%some commonly used underlined and
%hated symbols
\newcommand{\uY}{\ushort{\mbf{Y}}}
\newcommand{\ueY}{\ushort{Y}}
\newcommand{\uy}{\ushort{\mbf{y}}}
\newcommand{\uey}{\ushort{y}}
\newcommand{\ux}{\ushort{\mbf{x}}}
\newcommand{\uex}{\ushort{x}}
\newcommand{\uhx}{\ushort{\mbf{\hat{x}}}}
\newcommand{\uehx}{\ushort{\hat{x}}}

% Brackets
\newcommand{\br}[1]{{\left( #1 \right)}}
\newcommand{\sqbr}[1]{{\left[ #1 \right]}}
\newcommand{\cubr}[1]{{\left\{ #1 \right\}}}
\newcommand{\abr}[1]{\left< #1 \right>}
\newcommand{\abs}[1]{\left\vert #1 \right\vert}
\newcommand{\sabs}[1]{\vert #1 \vert}
\newcommand{\babs}[1]{\big\vert #1 \big\vert}
\newcommand{\sfloor}[1]{{\lfloor #1 \rfloor}}
\newcommand{\floor}[1]{{\left\lfloor #1 \right\rfloor}}
\newcommand{\ceiling}[1]{{\left\lceil #1 \right\rceil}}
\newcommand{\ceil}[1]{\left\lceil #1 \right\rceil}
\newcommand{\round}[1]{{\left\lfloor #1 \right\rceil}}
\newcommand{\magn}[1]{\left\| #1 \right\|}
\newcommand{\fracpart}[1]{\left\langle #1 \right\rangle}
\newcommand{\sfracpart}[1]{\langle #1 \rangle}


% Referencing
\newcommand{\refeqn}[1]{\eqref{#1}}
\newcommand{\reffig}[1]{Figure~\ref{#1}}
\newcommand{\reftable}[1]{Table~\ref{#1}}
\newcommand{\refsec}[1]{Section~\ref{#1}}
\newcommand{\refappendix}[1]{Appendix~\ref{#1}}
\newcommand{\refchapter}[1]{Chapter~\ref{#1}}

\newcommand {\figwidth} {100mm}
\newcommand {\Ref}[1] {Reference~\cite{#1}}
\newcommand {\Sec}[1] {Section~\ref{#1}}
\newcommand {\App}[1] {Appendix~\ref{#1}}
\newcommand {\Chap}[1] {Chapter~\ref{#1}}
\newcommand {\Lem}[1] {Lemma~\ref{#1}}
\newcommand {\Thm}[1] {Theorem~\ref{#1}}
\newcommand {\Cor}[1] {Corollary~\ref{#1}}
\newcommand {\Alg}[1] {Algorithm~\ref{#1}}
\newcommand {\etal} {\emph{~et~al.}}
\newcommand {\bul} {$\bullet$ }   % bullet
\newcommand {\fig}[1] {Figure~\ref{#1}}   % references Figure x
\newcommand {\imp} {$\Rightarrow$}   % implication symbol (default)
\newcommand {\impt} {$\Rightarrow$}   % implication symbol (text mode)
\newcommand {\impm} {\Rightarrow}   % implication symbol (math mode)
\newcommand {\vect}[1] {\mathbf{#1}} 
\newcommand {\hvect}[1] {\hat{\mathbf{#1}}}
\newcommand {\del} {\partial}
\newcommand {\eqn}[1] {Equation~(\ref{#1})} 
\newcommand {\tab}[1] {Table~\ref{#1}} % references Table x
\newcommand {\half} {\frac{1}{2}} 
\newcommand {\ten}[1] {\times10^{#1}}
\newcommand {\bra}[2] {\mbox{}_{#2}\langle #1 |}
\newcommand {\ket}[2] {| #1 \rangle_{#2}}
\newcommand {\Bra}[2] {\mbox{}_{#2}\left.\left\langle #1 \right.\right|}
\newcommand {\Ket}[2] {\left.\left| #1 \right.\right\rangle_{#2}}
\newcommand {\im} {\mathrm{Im}}
\newcommand {\re} {\mathrm{Re}}
\newcommand {\braket}[4] {\mbox{}_{#3}\langle #1 | #2 \rangle_{#4}} 
\newcommand{\dotprod}[2]{ \left\langle #1 , #2 \right\rangle}
\newcommand {\trace}[1] {\text{tr}\left(#1\right)}

% spell things correctly
\newenvironment{centre}{\begin{center}}{\end{center}}
\newenvironment{itemise}{\begin{itemize}}{\end{itemize}}

%%%%% set up the bibliography style
\bibliographystyle{IEEEbib}
%\bibliographystyle{uqthesis}  % uqthesis bibliography style file, made
			      % with makebst

%%%%% optional packages
\usepackage[square,comma,numbers,sort&compress]{natbib}
		% this is the natural sciences bibliography citation
		% style package.  The options here give citations in
		% the text as numbers in square brackets, separated by
		% commas, citations sorted and consecutive citations
		% compressed 
		% output example: [1,4,12-15]

%\usepackage{cite}		
			
\usepackage{units}
	%nice looking units
		
\usepackage{booktabs}
		%creates nice looking tables
		
\usepackage{ifpdf}
\ifpdf
  \usepackage[pdftex]{graphicx}
  %\usepackage{thumbpdf}
  %\usepackage[naturalnames]{hyperref}
\else
	\usepackage{graphicx}% standard graphics package for inclusion of
		      % images and eps files into LaTeX document
\fi

\usepackage{amsmath,amsfonts,amssymb, amsthm, bm} % this is handy for mathematicians and physicists
			      % see http://www.ams.org/tex/amslatex.html

		 
\usepackage[vlined, linesnumbered]{algorithm2e}
	%algorithm writing package
	
\usepackage{mathrsfs}
%fancy math script

%\usepackage{ushort}
%enable good underlining in math mode

%------------------------------------------------------------
% Theorem like environments
%
\newtheorem{theorem}{Theorem}
%\theoremstyle{plain}
\newtheorem{acknowledgement}{Acknowledgement}
%\newtheorem{algorithm}{Algorithm}
\newtheorem{axiom}{Axiom}
\newtheorem{case}{Case}
\newtheorem{claim}{Claim}
\newtheorem{conclusion}{Conclusion}
\newtheorem{condition}{Condition}
\newtheorem{conjecture}{Conjecture}
\newtheorem{corollary}{Corollary}
\newtheorem{criterion}{Criterion}
\newtheorem{definition}{Definition}
\newtheorem{example}{Example}
\newtheorem{exercise}{Exercise}
\newtheorem{lemma}{Lemma}
\newtheorem{notation}{Notation}
\newtheorem{problem}{Problem}
\newtheorem{proposition}{Proposition}
\newtheorem{property}{Property}
\newtheorem{remark}{Remark}
\newtheorem{solution}{Solution}
\newtheorem{summary}{Summary}
%\numberwithin{equation}{section}
%--------------------------------------------------------


\usepackage{tikz}
\usetikzlibrary{calc}
\usepackage{pgfplots}
\pgfplotsset{compat=1.8}
\usetikzlibrary{pgfplots.groupplots}

%\usepackage{xr}
%\externaldocument{paper2}

\usepackage{amsmath,amsfonts,amssymb, amsthm, bm}

\usepackage[square,comma,numbers,sort&compress]{natbib}

\newcommand{\sgn}{\operatorname{sgn}}
\newcommand{\sinc}{\operatorname{sinc}}
\newcommand{\rect}[1]{\operatorname{rect}\left(#1\right)}

%opening
\title{Fast sparse period estimation}
\author{R.~G.~McKilliam, I.~V.~L.~Clarkson and B.~G.~Quinn    
\thanks{
R.~G.~McKilliam is with the Institute for Telecommunications Research, The University of South Australia, SA, 5095.  I.~V.~L.~Clarkson is with the School of Information Technology \& Electrical Engineering, The University of Queensland, QLD., 4072, Australia.  B.~G.~Quinn is with the Department of Statistics, Macquarie University, Sydney, NSW, 2109, Australia.
}}

\begin{document}

\maketitle

\begin{abstract}

The problem of estimating the period of a point process from observations that are both sparse and noisy is considered.  By sparse it is meant that only a potentially small unknown subset of the process is observed.  By noisy it is meant that the subset that is observed, is observed with some error, or noise.  Existing accurate algorithms for estimating the period require $O(N^2)$ operations where $N$ is the number of observations.  By quantizing the observations we produce an estimator that requires only $O(N\log N)$ operations by use of the chirp z-transform or the fast Fourier transform.  The quantization has the adverse effect of decreasing the accuracy of the estimator.  This is investigated by Monte-Carlo simulation.  The simulations indicate that significant computational savings are possible with negligible loss in statistical accuracy.

\end{abstract} 
\begin{IEEEkeywords}
Period estimation, fast Fourier transform
\end{IEEEkeywords}

\section{Introduction}

Consider the periodic sequence of real numbers
\begin{equation}\label{eq:periodicprocess}
T_0 k + \theta_0, \qquad k \in \ints
\end{equation}
with period $T_0$ and phase $\theta_0$.  For example, the sequence might represent times at which the beat occurs in a musical score.  A problem of interest is to estimate $T_0$ and $\theta_0$ from $N$ observations $y_1,\dots,y_N$ of the form
\begin{equation} \label{eq:sigmodel}
y_n = T_0 s_n + \theta_0 + w_n, \qquad n = 1,\dots,N
\end{equation}
where $w_1,\dots,w_N$ represent noise and $s_1,\dots,s_N$ are unknown integers representing the subset of the periodic sequence that is observed.  The $y_1, \dots, y_N$ are \emph{sparse}, owing to the unknown integers $s_1,\dots,s_N$, and \emph{noisy}, owing to $w_1,\dots,w_N$, observations of the sequence~\eqref{eq:periodicprocess}.  For example, $y_1,\dots,y_N$ might represent times at which a bass drum hit is observed.  The drum hits might not occur at every beat and might occur off beat on some occasions.  Our aim in this case would be to estimate the beat of the music from the times of the drum hits.  The model is depicted in Figure~\ref{fig_stat_model}. %It is clear that any estimator of $T_0$ and $\theta_0$ must, either explicitly or implicitly, also produce and estimate of the unknow integers $s_1,\dots,s_N$ describing the subset of the periods that are observed.  

{
\def\vertgap{2}
\def\ph{0.4}
\def\T{1.1}
\newcommand{\raxis}{\draw[->] (-0.25,0) -- (8,0) node[above] {$\reals$}; \draw (0,-0.06)-- node[below] {$0$} (0,0.06) }
\newcommand{\pulse}[1]{ \draw[->,>=latex] (#1,0) -- (#1,1) }
\newcommand{\pulsewithnode}[2]{ \draw[->,>=latex] (#1,0) -- node[right] {#2} (#1,1) }
\begin{figure}[t]
	\centering
\begin{tikzpicture}[font=\small]
    %\draw[very thin,color=gray] (-0.1,-1.1) grid (3.9,3.9);
\begin{scope}
    \raxis;
    \foreach \k in {0,1,...,6} { \k, \pulse{\T*\k+\ph}; }
    \draw[<->] (\T*2+\ph,1.2) -- node[above] {$T_0$} (\T*3+\ph,1.2) ;
    \draw[<->] (0.0,1.2) -- node[above] {$\theta_0$} (\ph,1.2) ; 
%    \draw[->] (-0.25,-2*\vertgap) -- (8,-2*\vertgap) node[above] {$\reals$};
\end{scope}
\begin{scope}[yshift=-\vertgap cm]
  \raxis;
  \node[below] at (\T*1+\ph-0.3,0) {$y_1$}; \pulsewithnode{\T*1+\ph-0.3}{$s_1=1$}; \draw[->] (\T*1+\ph,1.2) -- node[above] {$w_1$} (\T*1+\ph-0.3,1.2) ; 
  \node[below] at (\T*3+\ph+0.5,0) {$y_2$}; \pulsewithnode{\T*3+\ph+0.5}{$s_2=3$}; \draw[->] (\T*3+\ph,1.2) -- node[above] {$w_2$} (\T*3+\ph+0.5,1.2) ;
  \node[below] at (\T*6+\ph-0.4,0) {$y_3$}; \pulsewithnode{\T*6+\ph-0.4}{$s_3=6$};\draw[->] (\T*6+\ph,1.2) -- node[above] {$w_3$} (\T*6+\ph-0.4,1.2) ;
\end{scope}
\end{tikzpicture} 
%		\includegraphics{figs/figures-1.mps}
		\caption{Sparse noisy observations of a periodic sequence.  The original sequence~\eqref{eq:periodicprocess} is shown on the top and sparse noisy observations according to~\eqref{eq:sigmodel} are shown on the bottom.  In this example the observed subset is represented by integers $s_1=1$, $s_2=3$ and $s_3=6$.}
		\label{fig_stat_model}
\end{figure}
}

The problem of estimating $T_0$ is called the \emph{sparse, noisy period estimation problem}~\cite{Clarkson2007,McKilliam2007}.  Solutions have applications to synchronisation for telecommunications devices~\cite{Fogel1988,Fogel1989_bit_synch_zero_crossings,Sidiropoulos2005,5621928}, estimation of the pulse repetition interval in electronic support~\cite{EltonGray_puilse_train_rader_1994,Gray_more_pri_1994,Clarkson_thesis,clarkson_estimate_period_pulse_train_1996,Hauochan_pri_2012}, and beat and tempo estimation in music~\cite{dixon_beat_extraction_2001}.  The problem is related to spike interval estimation in neurology~\cite{Arnett_neuro_pri_1976,Brillinger_spike_trains_1988,Rossoni200630} and a similar model has been proposed for the detection of anomalies, such as denial-of-service attacks, in internet traffic~\cite{He_detecting_periodic_patterns_in_internet_2006,5585849,5947313}.  The problem is also connected with classical problems in number theory such as Diophantine approximation, lattice reduction, and the study of prime numbers~\cite{Cassels_geom_numbers_1997,490557,Clarkson_thesis,Lenstra1982,Wubben_2011,536682,726812,CaseySadler_primes_2013}.
 
Least squares estimation of period has been studied in~\cite{Clarkson2007,McKilliam2007,Quinn_sparse_noisy_SSP_2012}.  The least squares estimator can be closely approximated using $O(N^2)$ operations using special purpose algorithms~\cite{McKilliam2009CoxeterLattices,McKilliam2008}.  The least squares estimator is also the maximum likelihood estimator in the case that the noise variables $w_1,\dots,w_N$ are independent and identically normally distributed.  An alternative estimator, called the~\emph{periodogram estimator}, was studied by Fogel and Gavish~\cite{Fogel1988,Fogel1989_bit_synch_zero_crossings} and involves the maximisation of Bartlett's periodogram for point processes~\cite{Bartlest_periodgram_point_process_1963}.  The statistical properties of the periodogram estimator were recently studied by Quinn~et.~al.~\cite{Quinn20013asilomar_period_est} who show the estimator to be strongly consistent and asymptotically normal under mild conditions on the noise $w_1,\dots,w_N$ and integers $s_1,\dots,s_N$.  The fastest existing algorithms can closely approximate the periodogram estimator in $O(N^2)$ operations.  We describe the periodogram estimator in Section~\ref{sec:peri-estim}.

In Section~\ref{sec:quant-peri} we show that, by quantizing the observations $y_1,\dots,y_N$, an approximation to the periodogram estimator can be computed using only $O(N\log N)$ operations.  This is facilitated by either the chirp z-transform~\cite{Rabiner1969} or the fast Fourier transform.  The quantization has the adverse affect of decreasing the accuracy of the estimator.  The estimator is accurate if a sufficiently fine quantizer is chosen, but this requires a larger transform.  The trade off between quantization level, accuracy, and computational complexity is investigated by computer simulation in Section~\ref{sec:simulations}.  The simulations indicate that significant computational savings are possible with negligible loss in accuracy.  Sadler and Casey~\cite{726812} have previously suggested use of the fast Fourier transform for period estimation.  However, their estimator requires far larger transforms and produces less accurate results than the estimator described here (see Section~VI and Figures 1-4 in~\cite{726812}).  In contrast, our estimator has accuracy similar to the least squares estimator, i.e., similar to the maximum likelihood estimator if the noise variables $w_1,\dots,w_N$ are assumed independent and identically normally distributed.
 
\section{The periodogram estimator}\label{sec:peri-estim}

%Before describing the periodogram estimator we state some required assumptions.  
%Observe that
%\[
%y_n = T_0 s_n + \theta_0 + w_n = \frac{T_0}{2} r_n + \theta_0 + w_n
%\]
%where $r_n = 2s_n \in \ints$.  
To ensure identifiability it is necessary to assume, as in~\cite{Sidiropoulos2005}, that $T_0$ lies in some interval $(T_{\text{min}}, T_{\text{max}})$ where $T_{\text{max}} < 2T_{\text{min}}$.%  We also assume that $y_1,\dots,y_N$ are in ascending order.  The periodogram estimator is unchanged by a permutation observations, and so, this assumption can be made without loss of generality because it amounts only to sorting the observations if required.
Fogel and Gavish~\cite{Fogel1988,Fogel1989_bit_synch_zero_crossings} proposed the estimator $\hat{T} = 1/\hat{f}$ where $\hat{f}$ is the maximiser of the~\emph{periodogram}
\[
I_y(f) = \babs{ \sum_{n=1}^N e^{ 2\pi j f y_n} }^2 = \babs{ \sum_{n=1}^N e^{ -2\pi j f y_n} }^2 
\]
over the interval $(f_{\text{min}}, f_{\text{max}}) = (1/T_{\text{max}}, 1/T_{\text{min}})$.  That is,
\[
\hat{f} = \arg\max_{f \in (f_{\text{min}}, f_{\text{max}})} I_y(f).
\] 
This is called the \emph{periodogram estimator}.  %This definition of the periodogram differs from the common definition in the spectral and frequency estimation literature~\cite{Hannan_time_series_1967,Quinn2001}.  
The statistical properties of the periodogram estimator were examined in~\cite{Quinn20013asilomar_period_est}.

The periodogram $I_y$ is not convex and has many local maxima.  In order to compute $\hat{f}$ it is usual to first compute $I_y$ on a grid of frequencies
\[
G = \{ f_{\text{min}} + Dk \mid k = 0, \dots K-1 \}
\]
where $D$ is the distance between frequencies in the grid, 
\[
K =  \floor{\frac{f_{\text{max}} - f_{\text{min}}}{D}}
\]
is the number of frequencies in the grid, and $\floor{\cdot}$ denotes the largest integer less than or equal to its argument.  
Let $\widetilde{f}$ be the maximiser of $I_y$ over $G$, that is, let
\[
\widetilde{f} = \arg\max_{f \in G} I_y(f).
\]
If the distance $D$ is sufficiently small, then $\widetilde{f}$ is a good approximation to the maximiser $\hat{f}$.  An iterative procedure, such as the Newton-Raphson method, or Brent's method~\cite[Ch.~5]{Brent_opt_no_derivs_1973} 
can then be used to compute $\hat{f}$ starting from $\widetilde{f}$.

\newcommand{\ymax}{y_{\text{max}}}
\newcommand{\ymin}{y_{\text{min}}}

In~\cite{McKilliam2007} it was observed that $D$ must shrink as $N$ increases in order for $\widetilde{f}$ to be a sufficiently good approximation to $\hat{f}$.  This was recently studied by Ye~et.~al.~\cite{Haohuan2013435,6492742} who heuristically suggest choosing
\begin{equation}\label{eq:gridwidthrecommended}
D = \frac{1}{2(\ymax - \ymin)}
\end{equation}
where $\ymax$ and $\ymin$ are the largest and smallest values from $y_1,\dots,y_N$.  This choice is supported by the simulations in~\cite{Haohuan2013435,6492742} and also those in Section~\ref{sec:simulations}.  With this choice $D = O(N^{-1})$ under reasonable assumptions about the integers $s_1,\dots,s_N$ and the noise $w_1,\dots,w_N$.  The asymptotic standard deviation of $\hat{f}$ is of order $O(N^{-3/2})$ as shown in~\cite{Quinn20013asilomar_period_est} and this initially suggests that $D = O(N^{-1})$ is too large.  However, the choice~\eqref{eq:gridwidthrecommended} does appear effective in practice.  A statistically rigorous justification for this choice of $D$ is beyond the scope of the present paper, but could potentially be developed using techniques like those in~\cite{Quinn2008maximizing_the_periodogram}.

With this choice for $D$ the number of elements in the grid grows linearly with $N$, that is, $K = O(N)$.  Computing $I_y(f)$ for each $f$ requires $O(N)$ operations and so computing $I_y(f)$ for all $f \in G$ requires $O(N^2)$ operations.  It follows that computation of $\widetilde{f}$ requires $O(N^2)$ operations and that the periodogram estimator $\hat{T} = 1/\hat{f}$ requires at least $O(N^2)$ operations to compute by this method.  %The choice for $D$ from~\eqref{eq:gridwidthrecommended} might not guarantee that application of a numerical optimisation proceedure, starting from $\widetilde{f}$, will result in $\hat{f}$.  For this reason, this approach can only be considered approximate.  However, the simulations presented in~\cite{Haohuan2013435,6492742} and those in Section~\ref{sec:simulations} suggest that~\eqref{eq:gridwidthrecommended} the approximation is very accurate.  
In the next section we describe a modification of the periodogram estimator.  The modification is such that the fast Fourier transform or the chirp z-transform~\cite{Rabiner1969} can be used to compute an approximation $\widetilde{f}^\prime$ to $\hat{f}$ with only $O(N\log N)$ operations.

\section{Quantization and the periodogram}\label{sec:quant-peri} 

Denote by $\round{x} = \sfloor{x + \tfrac{1}{2}}$ the nearest integer to $x$.  %The direction of rounding for half integers is not important.  Half integers are rounded up in our implementation.  
Let $q$ be a positive real number and define the quantization function
\[
Q(x) = \round{qx}/q
\]
that rounds its argument to a nearest multiple of $1/q$.  Large $q$ corresponds with fine quantization and small $q$ corresponds with coarse quantization.  Compute quantized observations
\[
z_n = Q(y_n) =  \ell_n/q, \qquad n = 1,\dots,N
\] 
where $\ell_n = \round{qy_n} \in \ints$.  The periodogram of the quantized observations $z_1,\dots,z_N$ is
\begin{align*}
I_z(f) = \babs{ \sum_{n=1}^N e^{ -2\pi j f z_n} }^2 = \babs{ \sum_{n=1}^N e^{ -2\pi j f \ell_n/q} }^2.
\end{align*}
Observe that $I_z$ is periodic with period $q$.  This indicates that quantization should be fine enough that 
\begin{equation}\label{eq:boundaliasingP}
q > f_{\text{max}} - f_{\text{min}}.
\end{equation}
Otherwise there would exist frequencies in the interval $(f_{\text{min}}, f_{\text{max}})$ that are aliases with respect to $I_z$.  %That is, there would exists two frequences $f_1,f_2 \in (f_{\text{min}}, f_{\text{max}})$ such that $f_1 = f_2 + 1/P$ and $I_z(f_1) = I_z(f_2)$.

Let $\widetilde{f}^\prime$ be the maximiser of $I_z$ over the grid $G$, that is,
\[
\widetilde{f}^\prime = \arg\max_{f \in G} I_z(f).
\]
If the quantization parameter $q$ is sufficiently large we expect that $I_z$ is close to $I_y$ and so $\widetilde{f}^\prime$ will be close to $\widetilde{f}$.  Thus, if $q$ is sufficiently large and the distance $D$ is sufficiently small we expect $\widetilde{f}^\prime$ to be a good approximation to $\hat{f}$.  An iterative procedure %or Brent's method~\cite{Brent_opt_no_derivs_1973} 
can then be used to compute $\hat{f}$ by maximising $I_y$ starting from $\widetilde{f}^\prime$.  

\newcommand{\ellmin}{\ell_{\text{min}}}
\newcommand{\ellmax}{\ell_{\text{max}}}

The maximiser $\widetilde{f}^\prime$ can be computed efficiently using the chirp z-transform.  Let $b_i$ be the number of integers from $\ell_1,\dots,\ell_N$ that are equal to $i$.  That is,
\[
b_i= \#\{ n \mid \ell_n = i, n = 1,\dots, N\}
\]
where $\#$ indicates the number of elements in a set.  Put $\ellmax = \round{q \ymax}$ and $\ellmin = \round{q \ymin}$ so that $b_i = 0$ if $i < \ellmin$ or $i > \ellmax$.  Now
\[
I_z(f) = \babs{ \sum_{i = \ellmin}^{\ellmax} e^{ -2\pi j f i/q} b_i }^2 = \babs{ \sum_{m = 0}^{\ellmax-\ellmin} e^{ -2\pi j f m/q} b_{m+\ellmin} }^2.
\]
We want to compute $I_z(f)$ for all $f$ in the grid $G$.  Putting
\[
v_m = e^{ -2\pi j f_{\text{min}}m/q} b_{m+\ellmin}
\]
we have
\[
I_z(f_{\text{min}} + Dk) = \babs{ \sum_{m = 0}^{\ellmax-\ellmin} e^{ -2\pi j Dk m/q} v_m }^2 = \abs{V_k}^2
\]   
where, using the notation of Rabiner~et.~al.~\cite[eq.~(10)]{Rabiner1969},
\[
V_k = \sum_{m = 0}^{\ellmax-\ellmin} e^{ -2\pi j Dk m/q} v_m = \sum_{m=0}^{\ellmax-\ellmin} A^{-m} W^{km} v_m,
\]
where $A = 1$, and $W = e^{ -2\pi j D/q}$.  All of $V_0,\dots,V_{K-1}$ can be efficiently computed by the chirp z-transform~\cite{Rabiner1969}.  Putting
\[
\hat{k} = \arg\max_{k = 0,1,\dots,K-1} \abs{ V_k }^2
\]
we have
\[ 
\widetilde{f}^\prime = f_{\text{min}} + D\hat{k} = \arg\max_{f \in G} I_z(f).
\]
%The frequency $\widetilde{f}^\prime$ can be used to initialize an iterative procedure to maximise $I_y$.  The procedure will converge to $\hat{f}$ if $\widetilde{f}^\prime$ is sufficiently close to $\hat{f}$.

An alternative to the chirp z-transform is to use a single fast Fourier transform.  We have found this to be faster than the chirp z-transform for all but exceptionally large $N$.  Suppose that we choose the distance $D$ and quantization parameter $q$ so that $M = q/D$ is an integer and
\[
M = \frac{q}{D} > \ellmax-\ellmin.
\]
With this choice
\[
I_z(f_{\text{min}} + Dk) = \sabs{\sum_{m = 0}^{M-1} e^{-2\pi j k m/M} v_m}^2 = \sabs{V_k}^2
\]
where $V_0,\dots,V_{M-1}$ are the Fourier coefficients of $v_0,\dots,v_{M-1}$ and can be computed using a fast Fourier transform of length $M$.  From inequality~\eqref{eq:boundaliasingP} we have
\[
M = \frac{q}{D} > \frac{f_{\text{max}} - f_{\text{min}}}{D} \geq \floor{\frac{f_{\text{max}} - f_{\text{min}}}{D}} = K
\]
and so, the required values  $V_0,\dots,V_{K-1}$ are the first $K$ Fourier coefficients.


\section{Computational complexity}\label{sec:comp-compl}

The previous section showed how $\widetilde{f}^\prime$ could be computed using either the chirp z-transform or a fast Fourier transform of length $M$.  We now describe recommended values for the parameters $q, M$, and $D$.  The simulations presented in Section~\ref{sec:simulations} suggest that a reasonable value for the quantization parameter is $q = 5 f_{\text{max}}$.  If the chirp z-transform is used then $D$ is chosen according to~\eqref{eq:gridwidthrecommended}.  Computation of the chirp z-transform requires the computation of three fast Fourier transforms, each of length $L + K - 1$, where 
\[
L = \ellmax - \ellmin + 1 = \round{q\ymax} - \round{q\ymin} + 1.
\]
If instead a single fast Fourier transform is used then our recommended value for the length of the transform is $M= 2L$.  In this case the grid distance
\[ 
D = \frac{q}{M} = \frac{q}{2(\round{q\ymax} - \round{q\ymax} + 1)} \approx  \frac{1}{2(\ymax - \ymin)}
\]
is approximately equal to~\eqref{eq:gridwidthrecommended}.  

Observe that $L$ grows proportionally with $q$ and that $L = O(qN)$ under reasonable assumptions about the integers $s_1,\dots,s_N$ and the noise variables $w_1,\dots,w_N$.  The most computationally expensive part of the chirp z-transform is the computation of three fast Fourier transforms of length $L+K-1$.  Alternatively, a single fast Fourier transform of length $M = 2L$ can be used.  Both approaches require $O(qN\log(qN)) = O(N\log N)$ operations because $q$ does not depend on $N$.  However, it is apparent that the length of the transforms, and hence the complexity, increases as the quantization becomes finer, i.e., as $q$ increases.  This presents a trade off between computational complexity and quantization accuracy.  This trade off is investigated in the next section.


\begin{figure}[t]
  \centering 
  \begin{tikzpicture}
    \selectcolormodel{gray} 
    \begin{axis}[font=\footnotesize,xmode=log,ymode=log,height=7cm,width=9cm,xlabel={$N$},ylabel={time (ms)},ylabel style={at={(0.08,0.41)}},xlabel style={at={(0.5,0.1)}}, legend style={draw=none,fill=none,legend pos=north west,cells={anchor=west},font=\footnotesize}]
      \addplot[mark=o,mark options={scale=1}] table {code/data/PeriodogramBenchmark};
      \addplot[mark=x,mark options={solid,fill=black,scale=1.3}] table {code/data/LeastSquaresBenchmark};
      \addplot[mark=*,mark options={solid,fill=black,scale=0.5}] table {code/data/QuantisedPeriodogramq4.0Benchmark};
      \addplot[mark=triangle,mark options={solid,fill=black,scale=0.9}] table {code/data/QuantisedPeriodogramChirpZq4.0Benchmark};
      %\addplot[mark=square,color=black,mark options={solid,fill=black,scale=0.8}] table {code/data/SLS2novlpBenchmark};
      \addplot[mark=square,color=black,mark options={solid,fill=black,scale=0.8}] table {code/data/SLS2newBenchmark};
      \legend{Periodogram, Least squares, Fast Fourier transform, Chirp z-transform, SLS2-new}
   \end{axis}
  \end{tikzpicture}  
  \caption{Computation time in milliseconds versus $N$ for the periodogram, least squares, and quantized periodogram estimators computed using the chirp z-transform and a single fast Fourier transform.}\label{plot:benchmark}
\end{figure} 


\begin{figure*}[t] 
  \centering 
  \begin{tikzpicture} 
    \selectcolormodel{gray}
\begin{groupplot}[
    group style={
        group name=my plots,
        group size=3 by 3,
        ylabels at=edge left
    },
    legend style={
      draw=none,
      fill=none,
      % legend pos=north west,
      % cells={anchor=west},
      % font=\footnotesize
    },
    footnotesize,
    width=9cm,
    height=11cm,
    tickpos=left,
    ytick align=outside,
    xtick align=outside,
    xmode=log,
    ymode=log,
    enlarge x limits=false 
]

\nextgroupplot[
xlabel={$\sigma^2$},
ylabel={MSE},
]
 
\addplot[mark=none,color=black] table {code/data/PeriodogramCLTN30geom1}; 
\addplot[mark=none,color=black,dashed] table {code/data/NormalisedLLSCLTN30geom1};
\addplot[mark=triangle,only marks,mark options={solid,fill=black,scale=1}] table {code/data/QuantisedPeriodogramFFTN30q1.5geom1};
\addplot[mark=*,only marks,color=black,mark options={solid,fill=black,scale=0.5}] table {code/data/QuantisedPeriodogramFFTN30q5.0geom1};
\addplot[mark=o,only marks,color=black,mark options={solid,fill=black,scale=1}] table {code/data/PeriodogramN30geom1};
\addplot[mark=x,only marks,color=black,mark options={solid,fill=black,scale=1.2}] table {code/data/NormalisedSamplingLLSN30geom1}; 
% \addplot[mark=square,only marks,color=black,mark options={solid,fill=black,scale=0.8}] table {code/data/SLS2novlpN30geom1};
\addplot[mark=square,only marks,color=black,mark options={solid,fill=black,scale=1}] table {code/data/SLS2newN30geom1};
% \addplot[mark=none,dashed,mark options={solid,fill=black,scale=1}] table {code/data/QuantisedPeriodogramFFTN30q2};
% \addplot[mark=none,color=black] table {code/data/PeriodogramCLTN200};
% \addplot[mark=none,color=black,dashed] table {code/data/NormalisedLLSCLTN200}; 
% \addplot[mark=*,only marks,color=black,mark options={solid,fill=black,scale=0.5}] table {code/data/PeriodogramN200};
% \addplot[mark=x,only marks,color=black,mark options={solid,fill=black,scale=1}] table {code/data/QuantisedPeriodogramFFTN200q3};
% \addplot[mark=o,only marks,color=black,mark options={solid,fill=black,scale=1}] table {code/data/QuantisedPeriodogramFFTN200q10};
% \addplot[mark=x,only marks,color=black,mark options={solid,fill=black,scale=0.8}] table {code/data/NormalisedSamplingLLSN200};
\addplot[color=gray,mark=none,color=black] table {code/data/PeriodogramCLTN1200geom1}; 
\addplot[color=gray,mark=none,color=black,dashed] table {code/data/NormalisedLLSCLTN1200geom1};
\addplot[color=gray,mark=o,mark options={color=black,solid,fill=black,scale=1}] table {code/data/PeriodogramN1200geom1};
\addplot[color=gray,mark=x,mark options={color=black,solid,fill=black,scale=1.2}] table {code/data/NormalisedSamplingLLSN1200geom1};
% \addplot[mark=square,only marks,color=black,mark options={solid,fill=black,scale=0.8}] table {code/data/SLS2novlpN1200geom1};
\addplot[color=gray,mark=square,mark options={solid,color=black,fill=black,scale=1}] table {code/data/SLS2newN1200geom1};
\addplot[color=gray,mark=triangle,mark options={solid,color=black,fill=black,scale=1}] table {code/data/QuantisedPeriodogramFFTN1200q1.5geom1};
% \addplot[mark=x,only marks,color=black,mark options={solid,fill=black,scale=1}] table {code/data/QuantisedPeriodogramFFTN1200q3};
\addplot[color=gray,mark=*,mark options={solid,color=black,fill=black,scale=0.5}] table {code/data/QuantisedPeriodogramFFTN1200q5.0geom1};
% \legend{Periodogram Theory, Least squares theory, Quantised periodogram, Periodogram, Least squares}
\legend{Periodogram theory, Least squares theory, $q=\tfrac{3}{2}f_{\text{max}}$, $q=5 f_{\text{max}}$, Periodogram, Least squares, SLS2-new}
\node[rotate=17,font=\footnotesize] at (axis cs:6e-4,1e-7) {$N=30$};
\node[rotate=17,font=\footnotesize] at (axis cs:7e-4,1.5e-12) {$N=1200$}; 

\nextgroupplot[
xlabel={$\sigma^2$}
]
      \addplot[mark=none,color=black] table {code/data/PeriodogramCLTN30geom10}; 
      \addplot[mark=none,color=black,dashed] table {code/data/NormalisedLLSCLTN30geom10}; 
            \addplot[mark=triangle,only marks,mark options={solid,fill=black,scale=1}] table {code/data/QuantisedPeriodogramFFTN30q1.5geom10};
      \addplot[mark=*,only marks,color=black,mark options={solid,fill=black,scale=0.5}] table {code/data/QuantisedPeriodogramFFTN30q5.0geom10};
      \addplot[mark=o,only marks,color=black,mark options={solid,fill=black,scale=1.1}] table {code/data/PeriodogramN30geom10};
       \addplot[mark=x,only marks,color=black,mark options={solid,fill=black,scale=1.3}] table {code/data/NormalisedSamplingLLSN30geom10}; 
  %     \addplot[mark=square,only marks,color=black,mark options={solid,fill=black,scale=0.8}] table {code/data/SLS2novlpN30geom10};
       \addplot[mark=square,only marks,color=black,mark options={solid,fill=black,scale=1}] table {code/data/SLS2newN30geom10};
%      \addplot[mark=none,dashed,mark options={solid,fill=black,scale=1}] table {code/data/QuantisedPeriodogramFFTN30q2};
 %     \addplot[mark=none,color=black] table {code/data/PeriodogramCLTN200};
 %     \addplot[mark=none,color=black,dashed] table {code/data/NormalisedLLSCLTN200}; 
 %     \addplot[mark=*,only marks,color=black,mark options={solid,fill=black,scale=0.5}] table {code/data/PeriodogramN200};
 %     \addplot[mark=x,only marks,color=black,mark options={solid,fill=black,scale=1}] table {code/data/QuantisedPeriodogramFFTN200q3};
 %     \addplot[mark=o,only marks,color=black,mark options={solid,fill=black,scale=1}] table {code/data/QuantisedPeriodogramFFTN200q10};
%      \addplot[mark=x,only marks,color=black,mark options={solid,fill=black,scale=0.8}] table {code/data/NormalisedSamplingLLSN200}; 
      \addplot[mark=none,color=black] table {code/data/PeriodogramCLTN1200geom10};
      \addplot[mark=none,color=black,dashed] table {code/data/NormalisedLLSCLTN1200geom10};
       \addplot[color=gray,mark=o,mark options={solid,color=black,fill=black,scale=1.1}] table {code/data/PeriodogramN1200geom10};
       \addplot[color=gray,mark=x,mark options={solid,color=black,fill=black,scale=1.3}] table {code/data/NormalisedSamplingLLSN1200geom10};
 %      \addplot[mark=square,only marks,color=black,mark options={solid,fill=black,scale=0.8}] table {code/data/SLS2novlpN1100geom10};
       \addplot[color=gray,mark=square,mark options={solid,color=black,fill=black,scale=1}] table {code/data/SLS2newN1200geom10};
       \addplot[color=gray,mark=triangle,mark options={solid,color=black,fill=black,scale=1}] table {code/data/QuantisedPeriodogramFFTN1200q1.5geom10};
      %\addplot[mark=x,only marks,color=black,mark options={solid,fill=black,scale=1}] table {code/data/QuantisedPeriodogramFFTN1100q3};
      \addplot[color=gray,mark=*,mark options={solid,color=black,fill=black,scale=0.5}] table {code/data/QuantisedPeriodogramFFTN1200q5.0geom10};
 %     \legend{Periodogram Theory, Least squares theory, Quantised periodogram, Periodogram, Least squares}
     %\legend{Periodogram theory, $q=\tfrac{3}{2}T_{\text{max}}$, $q=4T_{\text{max}}$, Periodogram, Least squares, SLS2-adj}
     \node[rotate=16,font=\footnotesize] at (axis cs:6e-4,9e-10) {$N=30$};
     \node[rotate=16,font=\footnotesize] at (axis cs:7e-4,1.5e-14) {$N=1200$}; 

\end{groupplot}

  \end{tikzpicture}  
  \caption{Mean square period error versus noise variance $\sigma^2$.  The integers $s_1,\dots,s_N$ are generated so that $s_1$ and $s_{n+1} - s_n$ for $n=1,\dots,N-1$ are independent and identically geometrically distributed with mean $\mu = 1$ in the plot on the left and mean $\mu=10$ in the plot on the right. Data points for $N=1200$ are connected by gray lines.}\label{plot:multipleN}
\end{figure*} 


\section{Simulations} \label{sec:simulations}

This section presents the results of Monte-Carlo simulations with the periodogram estimator~\cite{Haohuan2013435,Fogel1989_bit_synch_zero_crossings}, the modified least squares estimator~\cite{Clarkson2007,McKilliam2007,Quinn_sparse_noisy_SSP_2012}, Clarkson's modification of the separable least squares line search (SLS2-new) estimator~\cite{Clarkson2007,Sidiropoulos2005}, and the quantized periodogram estimator from Section~\ref{sec:quant-peri}.  The grid distance $D$ used for the periodogram and SLS2-new estimators is given by~\eqref{eq:gridwidthrecommended}.  The modified least squares estimator requires a smaller grid distance to work accurately and we choose~\eqref{eq:gridwidthrecommended} divided by three in our simulations.  %The quantized periodogram estimator is computed using a single fast Fourier transform of length $M = 2L = 2(\ellmax - \ellmin + 1)$.  
Both the periodogram and quanitised periodogram estimators use the Newton-Raphson method.  In all simulations the noise variables $w_1,\dots,w_N$ are independent and identically normally distributed with variance $\sigma^2$.  The integers $s_1,\dots,s_N$ are generated so that $s_1$ and $s_{n+1} - s_n$ for $n=1,\dots,N-1$ are independent and identically geometrically distributed starting from zero with means $\mu=1$ and $10$.  The true period is $T_0 = \pi/3$, the true phase is $\theta_0 = 0.2$, and $T_{\text{min}} = 0.8$ and $T_{\text{max}} = 1.5$.

Figure~\ref{plot:multipleN} shows the mean square error of the estimators versus $\sigma^2$ for $N = 30$ and $N = 1200$.  In the left hand  plot $\mu=1$ and in the right hand plot $\mu=10$.  %Thus, on average $\tfrac{1}{3}$ of the elements of the point process~\eqref{eq:periodicprocess} are observed in the left hand  plot and only $\tfrac{1}{15}$ in the right hand plot.  
Three thousand Monte-Carlo trials are run for each value of $\sigma^2$.  The plots show the mean square error of the quantized periodogram estimator for two quantization parameters $q = \tfrac{3}{2}f_{\text{max}} = \tfrac{15}{8}$ and $q = 5 f_{\text{max}} = \tfrac{25}{4}$.  The estimator performs poorly when $q = \tfrac{3}{2}f_{\text{max}}$ (triangles) but performs almost identically to the original periodogram estimator (circles) when $q = 5 f_{\text{max}}$ (dots).  The plot also shows the mean square error of the modified least squares estimator~\cite{Clarkson2007,McKilliam2007} and the SLS2-new estimator~\cite{Clarkson2007,Sidiropoulos2005}.  %The statistical performance of the SLS2-new estimator is poor when compared with the other estimators.  %For small $\sigma^2$ the mean square error of the SLS2 estimator is almost three orders of magnitude less accurate when $N = 50$ and almost five orders of magnitude less accurate when $N=1000$.    
The solid line shows the mean square error of the periodogram estimator predicted by Theorem~2 from~\cite{Quinn20013asilomar_period_est}.  The dashed line shows the mean square error of the modified least squares estimator predicted by Theorem~3 from~\cite{Quinn_sparse_noisy_SSP_2012}.

Figure~\ref{plot:benchmark} shows the average running time for the estimators for $N$ in the range $10$ to $25000$ when $\mu=3$.  Running times for other values of $\mu$ lead to similar conclusions.  The quantization parameter is $q = 5 f_{\text{max}}$ so that the performance of the quantized periodogram estimator is similar to that of the original periodogram estimator as observed in Figure~\ref{plot:multipleN}.  The average running times for the quantized periodogram estimator computed using the chirp z-transform and using a single fast Fourier transform are shown.  The chirp z-transform is slower in these simulations, but is expected to be faster when $N$ is exceptionally large.  %The quantized periodogram estimator computed using a single fast Fourier transform is faster than the periodogram estimator when $N \geq 20$.  It is faster than the least squares estimator when $N \geq 80$ and it is faster than the SLS2-new estimator when $N \geq 400$.  
The quantized periodogram estimator is significantly faster than the other estimators for large $N$.  For example, when $N = 25000$ the periodogram estimator requires approximately 530 seconds, whereas the quantized periodogram estimator requires approximately 1 second.  The software is written in Java and the computer used is an Intel Core2 Q6600 running at \unit[2.4]{GHz}. 

 

\section{Conclusion}

We have considered the problem of estimating the period of a point process from $N$ observations that are both sparse and noisy.  By quantizing the observations we describe an estimator that can be computed in $O(N\log N)$ operations by use of the fast Fourier transform or chirp z-transform.  This improved upon existing accurate estimators that require at least $O(N^2)$ operations.  The trade off between computational complexity and quantization accuracy was examined by simulation.  The simulations indicate that significant computational savings are possible with negligible loss in statistical accuracy.

The new estimator has potential application to the synchronisation of communications devices~\cite{Fogel1988,Fogel1989_bit_synch_zero_crossings}, pulse repetition interval estimation in electronic support~\cite{EltonGray_puilse_train_rader_1994,Gray_more_pri_1994,Clarkson_thesis,clarkson_estimate_period_pulse_train_1996,Hauochan_pri_2012}, beat and tempo estimation in music~\cite{dixon_beat_extraction_2001}, spike interval estimation in neurology~\cite{Arnett_neuro_pri_1976,Brillinger_spike_trains_1988,Rossoni200630}, and in the analysis of internet traffic~\cite{He_detecting_periodic_patterns_in_internet_2006,5585849,5947313}.


\small
\bibliography{bib}


\end{document}


%%% Local Variables:
%%% TeX-source-correlate-mode: t
%%% mode: latex
%%% TeX-master: t
%%% End: 


}

\end{document}





















